\documentclass[t,compress,xcolor=table]{beamer}
\mode<presentation>
\usetheme{Oxygen}
\setbeamertemplate{mini frames}[box] \usefonttheme[%
% % 	onlysmall, % headline, footline, and sidebars is changed
	onlylarge, % main title, frame titles, and section
]{structurebold}

\setbeamerfont*{frametitle}{size=\normalsize,series=\bfseries}

\hypersetup{pdfpagemode=FullScreen}

\usepackage[brazilian]{babel}
\usepackage{times}
\usepackage[T1]{fontenc}
\usepackage{hyperref}
\usefonttheme{serif}
\usepackage{fontspec}
\setsansfont{Bauhausb.ttf} 
\setmainfont{Verdana.ttf}
\usepackage{caption}
\usepackage{svg}
\usepackage{amsmath}
\usepackage{graphics}
\usepackage{graphicx}
\usepackage{pgf}			
\usepackage{multirow}	
\usepackage{tabularx}
\usepackage{wasysym}
\usepackage{lscape}
 \usepackage{booktabs}
\graphicspath{{./img/}}
\usepackage{graphicx}
% \usepackage[table,xcdraw]{xcolor}
%Comando de Subitem
\newcommand{\SubItem}[1]{
    {\setlength\itemindent{15pt} \item[-] #1}
}

% Setup TikZ
\usepackage{tikz}
\usetikzlibrary{arrows}
\tikzstyle{block}=[draw opacity=0.7,line width=1.4cm]

\newcommand{\plets}{$\textrm{PL}e\textrm{T}s$}

%\title[short title]{long title}
%\subtitle[short subtitle]{long subtitle}
%\author[short name]{long name}
%\date[short date]{long date}
%\institution[short name]{long name}

% Author, Title, etc.
\title[Extensionly]
{{\sffamily 
	Extensionly - Uma ferramenta de apoio à gestão de projetos e programas de extensão na universidade: Front end
}}

\author[Lucas Alexandre Fell]
{
	Lucas Alexandre Fell\inst{1} Maicon Bernardino\inst{1}
}

\date[Aug, 2022]{Alegrete, Brasil, 11/08/2022}

\institute[]
{
	\emph{lucasfell.aluno@unipampa.edu.br}\\
	\emph{bernardino@unipampa.edu.br}\\ 
	\vspace{7pt}
	\inst{1} Universidade Federal do Pampa (Unipampa)\\
	}

%\setbeamertemplate{footline}
%{
%	\leavevmode%
%	\hbox{%
%	
%	\begin{beamercolorbox}[wd=.25\paperwidth,ht=2.25ex,dp=1ex,center]{author in head/foot}%
%		%\usebeamerfont{author in head/foot}\insertshortauthor%~~(\insertshortinstitute)
%		\usebeamerfont{author in head/foot} iiWAS - 2013, Vienna, Austria
%		%Elder M. Rodrigues%~~(\insertshortinstitute)
%	\end{beamercolorbox}%
%	
%	\begin{beamercolorbox}[wd=.65\paperwidth,ht=2.25ex,dp=1ex,center]{title in head/foot}%
%		\usebeamerfont{title in head/foot}\inserttitle
%	\end{beamercolorbox}%
%	
%	\begin{beamercolorbox}[wd=.1\paperwidth,ht=2.25ex,dp=1ex,right]{date in head/foot}%
%		\usebeamerfont{date in head/foot} %\insertshortdate{}\hspace*{2em}
%		\insertframenumber{} / \inserttotalframenumber\hspace*{2ex}
%	\end{beamercolorbox}}%
%	
%	\vskip0pt%
%}

%####################################################################################

% The main document


\begin{document}

\begin{frame}[plain,t]
  \titlepage
\end{frame}

\begin{frame}{{\sffamily Sumário}}

  \begin{block}{}
    \begin{itemize}
      \item Introdução
      \item Metodologia Científica
      \item Embasamento Teórico
      \item Estudos Conduzidos
            \SubItem{Revisão na Literatura Cinza}
            \SubItem{Levantamento (\textit{survey})}
      \item Análise e Projeto do MVP
      \item Considerações Preliminares
    \end{itemize}
  \end{block}
\end{frame}


\section{Introdução}
\subsection*{Introdução}

\begin{frame}{{\sffamily Motivação}}
  \begin{block}{}
    \begin{itemize}%[<+->]
      \item Impactos da curricularização % Aumento da carga horaria de extensão, mais propostas e mais procura por atividades
            \SubItem{Aumento da carga horaria de extensão}
            \SubItem{Mais procura por ações extensionistas}
      \item Automação de processos da extensão
            \SubItem{Submissão e aprovação de propostas}
            \SubItem{Divulgação e envio de emails}
            \SubItem{Geração de certificados}
      \item Vínculo entre comunidade externa e acadêmica
            \SubItem{Geração de demandas}

    \end{itemize}
  \end{block}
\end{frame}

\begin{frame}{{\sffamily Questão de Pesquisa}}
  \begin{block}{Questão de Pesquisa}
    “Como uma ferramenta de apoio à gestão de programas e projetos de extensão da UNIPAMPA pode otimizar a gestão dos processos de proposição, registro, divulgação e responsabilidades das ações de extensão?”
  \end{block}
\end{frame}

\begin{frame}{{\sffamily Objetivos}}
  \begin{block}{Objetivo Geral}
    Desenvolver o frontend da ferramenta de apoio à gestão de programas e projetos de extensão da UNIPAMPA
  \end{block}
\end{frame}

\begin{frame}{{\sffamily Objetivos}}
  \begin{block}{Objetivos Específicos}
    \begin{itemize}
      \item Realizar revisão sistemática na literatura cinza
            \SubItem{Analisar ferramentas similares}
      \item Realizar levantamento (survey) com possíveis usuários
      \item Coletar e refinar requisitos para desenvolver a ferramenta
      \item Executar estudo de caso real para validar a ferramenta e coletar feedback
    \end{itemize}
  \end{block}
\end{frame}

\begin{frame}{{\sffamily Contribuição}}
  \begin{block}{}
    \begin{itemize}
      \item Revisão sistemática na literatura cinza
      \item Levantamento com possíveis usuários
      \item Estabelecimento de tecnologias para
            implementação do Minimum Viable Product (MVP)
      \item Desenvolvimento do MVP
    \end{itemize}
  \end{block}
\end{frame}

% Metodologia 3-4 pags, desenho da pesquisa (bpmn) com fases, atividades, cronograma desde o anteprojeto
%=======================================
\chapter{Methodology}\label{methodology}
%=======================================

This chapter discusses how the study was planned, the adopted methodology and the approaches used to conduct it. The next sections will describe in more detail the procedures and techniques used on the research. Scientific research is described on \Cref{sec:met-1}. In \Cref{sec:met-2}, the research classifications according to \citeonline{Prodanov:2013} are defined. After that, in \Cref{sec:met-3}, the research design is shown and explained. A research schedule was created and can be seen in \Cref{sec:met-schedule}. Finally, in \Cref{sec:met-4}, the whole chapter is briefly summarized.

\section{Introduction}\label{sec:met-1}

The word ``Science'' comes from the latin word ``Scire'', which means to learn and to know. For science to be done, there has to be a way to gather new information, building upon what is already known. This is where scientific research fits in. The scientific method, says \citeonline{Prodanov:2013}, is a way, through a set of adopted procedures, to achieve knowledge.

It is the basic instrument which turns thoughts into systems, ordering them through procedures, which guides the scientist along the way to achieve his predefined scientific goals. \citeonline{Prodanov:2013} also mentions that without the scientific method, there is no science.

\section{Research Classification}\label{sec:met-2}

This research study is defined according to the classification created by \citeonline{Prodanov:2013}. It has multiple research types, each of which can be classified into several categories according to the nature, goals, approach and procedures of the study. \Cref{fig:research-classification} shows how the research is categorized. The darker boxes represent categories which apply to this work. The terms in them are described in this section. The other boxes are kept for consistency with the original model.

\begin{figure}[!htb]
  \caption{Research Classification}\label{fig:research-classification}
  \begin{center}
    \includegraphics[width=16cm]{img/2-Research Classification@2x.png}
  \end{center}
  \fonte{Adapted from \cite{Prodanov:2013}.}
\end{figure}

Looking through the nature point of view, this is an \textbf{Applied Research}. It has the goal of generating knowledge to the solution of specific problems, through a practical application. It is related to local interests and often has a new process or product as a result.

From the objectives point of view, it is classified as an \textbf{Exploratory Research}, since one of its goals is to discover more information about what is being investigated, and maybe finding a new type of approach to the subject. This type of research generally takes the form of bibliographic research and \textbf{Case Studies}. The former doesn't apply to this study, though, because the final product won't be heavily inspired on white literature. Only the latter applies, because researches of this nature are more focused on the immediate application of knowledge in a circumstantial reality, emphasizing the development of theories.

However, the product will certainly be inspired by grey literature, meaning it fits as a \textbf{Documentary Research}. It is similar to bibliographic research, but the main difference between them is the nature of their sources. While bibliographic research makes fundamental use of contributions from various authors on a given subject, documentary research is based on materials that have not yet received an analytical treatment or that can be reworked according to the research objectives.

According to the technical procedures, this research also features a \textbf{Survey}. They are much more suitable for descriptive rather than explanatory studies. They are inappropriate for the deepening of more complex psychological and psychosocial aspects, but very effective for less delicate problems, for example, electoral preference and consumer behavior. The latter is much more aligned with this study than the former. Surveys are very useful for the study of opinions and attitudes, but little indicated in the study of problems referring to complex social structures. How this technique was applied in the scope of this work is described in detail in \Cref{survey}.

Through the approach point of view, the research is both \textbf{Quantitative}, meaning translating opinions and information into numbers to classify and analyze them. And also \textbf{Qualitative}, because some parts of the study can't be quantified, and must be understood subjectively. An example would be to receive written, detailed feedback from a target-user through the survey.

\section{Research Design}\label{sec:met-3}

In order to conduct the study correctly, a research design was created. The activities are grouped in five phases:
\begin{inparaenum}[(1)]
  \item gather information;
  \item begin development;
  \item write term paper;
  \item develop;
  \item evaluate.
\end{inparaenum}
They are all described in this section and can also be observed in \Cref{fig:research-design}.

\begin{figure}[!htb]
  \caption{Research Design}\label{fig:research-design}
  \begin{center}
    \includegraphics[width=16cm]{img/2-research design.pdf}
  \end{center}
  \fonte{Author.}
\end{figure}

The \textbf{gather information} group aims to create two tangible artifacts: the grey literature systematic review and the survey to better understand the scope of the goal product and most importantly collect a list of well defined requirements.

The \textbf{begin development} group is where the implementation and the term paper writing begins. This is where the technologies used throughout the development of the product are defined. The most important requirements should already be implemented as well.

Next, there is the \textbf{write term paper} group, in which both first and second term papers are going to be written and defended. It is important to notice that the first work will be written while the initial \ac{MVP} implementation is on going.

Continuing to the next milestone, is the \textbf{develop} group, where it is planned to finish the product development. After that, in the \textbf{evaluate} group, is where the real use case will be ran, and the feedback from it, analyzed. If all goes well, the product might turn into a real solution, adopted by the university to be used.

\section{Research Schedule}\label{sec:met-schedule}

In order to have a clear vision of the steps required to run this study, a timeline was created describing what will be done by month until the expected ending of the research. Refer to \Cref{tbl:schedule} for the full overview of what was planned.

\begin{table}
  \centering
  \caption{Research Schedule}
  \label{tbl:schedule}
  \scriptsize
  \begin{tabular}{p{3cm}|l|llll|lll|ll}
    \bottomrule
    \rowcolor[rgb]{0.753,0.753,0.753} \multicolumn{1}{c|}{{\cellcolor[rgb]{0.753,0.753,0.753}}}                                       & \multicolumn{1}{c|}{\textbf{2021/2}} & \multicolumn{4}{c|}{\textbf{2022/1}} & \multicolumn{3}{c|}{\textbf{2022/2}} & \multicolumn{2}{c}{\textbf{2023/1}}                                                                                                                                                                                                                                            \\
    \hhline{>{\arrayrulecolor[rgb]{0.753,0.753,0.753}}->{\arrayrulecolor{black}}----------}
    \rowcolor[rgb]{0.753,0.753,0.753} \multicolumn{1}{c|}{\multirow{-2}{*}{{\cellcolor[rgb]{0.753,0.753,0.753}}\textbf{ Activities}}} & \textbf{Nov - Mar}                   & \multicolumn{1}{c}{\textbf{Apr}}     & \textbf{May}                         & \textbf{Jun}                         & \textbf{Jul}                         & \textbf{Aug}                         & \textbf{Sep - Nov}                   & \textbf{Dec}                         & \multicolumn{1}{l|}{\textbf{Jan}}    & \textbf{Feb}                         \\
    \hline
    \rowcolor[rgb]{0.914,0.914,0.914} Plan and execute systematic review in the gray literature                                       & {\cellcolor[rgb]{0.753,0.753,0.753}} &                                      &                                      &                                      &                                      &                                      &                                      &                                      &                                      &                                      \\
    Plan and execute survey with target users                                                                                         &                                      & {\cellcolor[rgb]{0.753,0.753,0.753}} &                                      &                                      &                                      &                                      &                                      &                                      &                                      &                                      \\
    \rowcolor[rgb]{0.914,0.914,0.914} Analyze results from previous steps and map requirements                                        &                                      & {\cellcolor[rgb]{0.753,0.753,0.753}} & {\cellcolor[rgb]{0.753,0.753,0.753}} &                                      &                                      &                                      &                                      &                                      &                                      &                                      \\
    Plan and start tool development                                                                                                   &                                      &                                      & {\cellcolor[rgb]{0.753,0.753,0.753}} & {\cellcolor[rgb]{0.753,0.753,0.753}} &                                      &                                      &                                      &                                      &                                      &                                      \\
    \rowcolor[rgb]{0.914,0.914,0.914} Write Term Paper I                                                                              &                                      &                                      &                                      & {\cellcolor[rgb]{0.753,0.753,0.753}} & {\cellcolor[rgb]{0.753,0.753,0.753}} & {\cellcolor[rgb]{0.753,0.753,0.753}} &                                      &                                      &                                      &                                      \\
    Defend Term Paper I                                                                                                               &                                      &                                      &                                      &                                      &                                      & {\cellcolor[rgb]{0.753,0.753,0.753}} &                                      &                                      &                                      &                                      \\
    \rowcolor[rgb]{0.914,0.914,0.914} Continue the development of the tool                                                            &                                      &                                      &                                      &                                      &                                      &                                      & {\cellcolor[rgb]{0.753,0.753,0.753}} & {\cellcolor[rgb]{0.753,0.753,0.753}} &                                      &                                      \\
    Execute a real use case on the tool                                                                                               &                                      &                                      &                                      &                                      &                                      &                                      &                                      & {\cellcolor[rgb]{0.753,0.753,0.753}} & {\cellcolor[rgb]{0.753,0.753,0.753}} &                                      \\
    \rowcolor[rgb]{0.914,0.914,0.914} Write Term Paper II                                                                             &                                      &                                      &                                      &                                      &                                      &                                      &                                      & {\cellcolor[rgb]{0.753,0.753,0.753}} & {\cellcolor[rgb]{0.753,0.753,0.753}} & {\cellcolor[rgb]{0.753,0.753,0.753}} \\
    Defend Term Paper II                                                                                                              &                                      &                                      &                                      &                                      &                                      &                                      &                                      &                                      &                                      & {\cellcolor[rgb]{0.753,0.753,0.753}} \\
    \toprule
  \end{tabular}
  \fonte{Author.}
\end{table}

\section{Chapter Summary}\label{sec:met-4}

This chapter provided an idea of how the methodology is defined for the study and how the research can be classified. In addition, the created research design was presented, showcasing the different planned processes for the future and those that have already been executed. \Cref{background} describes all the information and background necessary for the success of this work, while also assisting the reader in better understanding the research methodology previously described.

% %------------------------------------------------------------------------------
% \section{Citações}
% %------------------------------------------------------------------------------

% \index{citações!diretas}Utilize o ambiente \texttt{citacao} para incluir citações diretas com mais de três linhas:

% \begin{citacao}
% As citações diretas, no texto, com mais de três linhas, devem ser destacadas com recuo de 4 cm da margem esquerda, com letra menor que a do texto utilizado e sem as aspas.
%   No caso de documentos datilografados, deve-se observar apenas o recuo \cite[5.3]{NBR10520:2002}
% \end{citacao}

% \index{citações!simples}Citações simples, com até três linhas, devem ser incluídas com aspas.
%   Observe que em \LaTeX~as aspas iniciais são diferentes das finais: ``Amor é fogo que arde sem se ver''.

% Para as citações indiretas, o comando padrão, \verb|\cite|, realiza a forma mais comum de citação \cite{SisbiUnipampa2011}.
%   A outra das formas mais usadas, para citar em texto corrido, é conseguida com o comando \verb|\citeonline|: segundo \citeonline{SisbiUnipampa2011}, na citação indireta, o número da página é opcional.


% %------------------------------------------------------------------------------
% \subsection{Referências internas}\label{sec:referencias_internas}
% %------------------------------------------------------------------------------

% Usa-se o comando \verb|\ref{}| para referenciar uma Tabela ou Figura.
%   Por exemplo, esta é uma referência para a Tabela~\ref{tab:nivinv}.
%   Mas também pode-se usar o comando \verb|\Cref{}|, que insere o tipo também.
%   Por exemplo, esta é outra referência para a \Cref{tab:nivinv}.

% Há vários outros comandos interessantes.
%   Eles estão no fonte do \Cref{introducao}, na \Cref{sec:referencias_internas}
%   \footnote{O número do capítulo indicado é \ref{introducao}, que se inicia à página \pageref{introducao}.}
%   (\nameref{introducao}, \autopageref{introducao}).


% %------------------------------------------------------------------------------
% \section{Tabelas}
% %------------------------------------------------------------------------------

% \index{tabelas}A \Cref{tab:nivinv} é um exemplo de tabela construída em \LaTeX.
%   Como sugestão de formatação, evite ao máximo o uso de linhas verticais.
%   As colunas de uma tabela devem ser separadas visivelmente.
%   O contrário indica que a tabela está mal formatada ou que certas informações não deveriam estar nela.

% Da mesma forma, evite o uso de linhas horizontais para separar linhas da tabela.
%   Use-as apenas para separar o cabeçalho e eventuais partes importantes.
%   Para obter um resultado ainda mais elegante, use os comandos do pacote \texttt{booktabs}.

% Veja essas sugestões aplicas na \Cref{tab:nivinv}.

% \begin{table}[!htb][!htb]
%   \footnotesize
%   \caption[Níveis de investigação]{Níveis de investigação.}
%   \label{tab:nivinv}
%   \begin{tabular}{m{2.6cm}m{6.0cm}m{2.25cm}m{3.40cm}}
%     \toprule
%     \textbf{Nível de Investigação} & \textbf{Insumos}                                                   & \textbf{Sistemas de Investigação} & \textbf{Produtos}    \\
%     \midrule
%     Meta-nível                     & Filosofia\index{filosofia} da Ciência                              & Epistemologia                     & Paradigma            \\
%     Nível do objeto                & Paradigmas do metanível e evidências do nível inferior             & Ciência                           & Teorias e modelos    \\
%     Nível inferior                 & Modelos e métodos do nível do objeto e problemas do nível inferior & Prática                           & Solução de problemas \\
%     \bottomrule
%   \end{tabular}
%   \fonte{\citeonline{van86}}
% \end{table}


% Uma opção avançada para a criação de tabelas é usar o pacote \texttt{pgfplotstable}.
%   Ele permite que os dados de um arquivo sejam lidos e colocados em uma tabela, formatando-os da maneira que se quiser.
%   A \Cref{tab:dados} é um exemplo.
%   Veja o arquivo \texttt{desenvolvimento.tex} para os comandos necessários.

% % Necessário o pacote filecontents
% % Especifica o conteúdo que será gravado no dado arquivo (nesse caso, resultados.txt)
% \begin{filecontents*}{resultados.txt}
% tamanho metodo1 metodo2 metodo3
% 10  30    36.2  28.3
% 20  54.8  52.5  56.8
% 30  65    59.6  74.1
% 40  64.5  59.6  76.7
% 50  64.6  59.6  76.5
% \end{filecontents*}

% % Para definir os estilos das colunas e da tabela
% \pgfplotstableset{
%      %columns={tamanho,metodo1,{grad(log(metodo2),log(metodo3))}},
%      columns/metodo1/.style={
%          column name=\textsc{Método 1 (\%)},
%          column type=c,
%          %dec sep align={c},
%          %sci,sci zerofill,sci subscript,
%          fixed,fixed zerofill,
%          precision=1},
%      columns/metodo2/.style={
%          column name=\textsc{Método 2 (\%)},
%          column type=c,
%          fixed,fixed zerofill,precision=1},
%      columns/metodo3/.style={
%          column name=\textsc{Método 3 (\%)},
%          fixed,fixed zerofill,precision=1},
%      columns/media/.style={
%          column name=\textsc{Média (\%)},
%          fixed,fixed zerofill,precision=1},
%      create on use/media/.style={
%          create col/expr={(\thisrow{metodo1}+\thisrow{metodo2}+\thisrow{metodo3})/3}},
%      every head row/.style={
%          before row=\toprule,after row=\midrule},
%      every last row/.style={
%          after row=\bottomrule}}

% \begin{table}[!htb][!phtb]
%   \caption{Exemplo de tabela com dados de arquivo.}
%   \label{tab:dados}
%   \begin{center}
%     % Lê do arquivo resultados.txt as colunas especificadas e as formata de acordo com
%     % os estilos acima ou com os estilos especificados aqui (nesse caso, para a coluna tamanho).
%     \pgfplotstabletypesetfile[
%       columns={tamanho,metodo1,metodo2,metodo3,media},
%       columns/tamanho/.style={column name=\textsc{Tamanho}}
%     ]{resultados.txt}
%   \end{center}
% \end{table}


% %------------------------------------------------------------------------------
% \section{Figuras}
% %------------------------------------------------------------------------------

% \index{figuras}Figuras podem ser criadas diretamente em \LaTeX.
%   Uma das melhores formas, por ser relativamente simples, bem documentada e gerar ótimos resultados, é com o uso do pacote tikz\footnote{Há vários exemplos em \url{http://www.texample.net/}.}.
%   Ele permite gerar diagramas, árvores, fluxogramas etc.
%   A \Cref{fig:fib} mostra um exemplo simples de árvore.

% \begin{figure}[!htb]
%   \caption{Árvore de recursão de Fibonacci.}\label{fig:fib}
%   \begin{center}
%   \begin{tikzpicture}[level/.style={sibling distance=160mm/(2^#1)},
%                       level 4/.style={sibling distance=18mm},
%                       every node/.style={minimum width=5mm}]
%     \node [circle,draw] (z) {$fib(5)$}
%       child {node [circle,draw] (a) {$fib(4)$}
%         child {node [circle,draw] (b) {$fib(3)$}
%           child {node [circle,draw=red] (c) {$fib(2)$}
%             child {node [circle,draw] (d) {$fib(1)$}}
%             child {node [circle,draw] (e) {$fib(0)$}}
%           }
%           child {node [circle,draw] (f) {$fib(1)$}}
%         }
%         child {node [circle,draw=red] (g) {$fib(2)$}
%           child {node [circle,draw] (h) {$fib(1)$}}
%           child {node [circle,draw] (i) {$fib(0)$}}
%         }
%       }
%       child {node [circle,draw] (j) {$fib(3)$}
%         child {node [circle,draw=red] (k) {$fib(2)$}
%           child {node [circle,draw] (l) {$fib(1)$}}
%           child {node [circle,draw] (m) {$fib(0)$}}
%         }
%         child {node [circle,draw] (n) {$fib(1)$}}
%       };
%   \end{tikzpicture}
%   \end{center}
% \end{figure}

% Junto com o pacote pgfplots também é possível gerar gráficos de funções ou a partir de dados em um arquivo (como no caso da \Cref{tab:dados}).
%   As Figuras \ref{fig:grafico1} e \ref{fig:grafico2} mostram exemplos de gráficos de função, e a \Cref{fig:grafico_dados} um exemplo de gráfico a partir dos mesmos dados que os da \Cref{tab:dados}.

% \begin{figure}[!htb]
%   \caption{Gráfico produzido diretamente no arquivo fonte.}\label{fig:grafico1}
%   \begin{center}
%   \begin{tikzpicture}[scale=1]
%   \begin{axis}[
%       width=.65\textwidth,
%       ymin=0,xmin=0,xmax=1000,
%       xlabel=$n$,ylabel=$T(n)$,
%       ylabel near ticks,
%       scaled ticks=false, % Evita o uso de notação exponencial 10^2
%       ticklabel style={/pgf/number format/.cd,fixed,use comma,1000 sep={}}, % Para vírgula como separador decimal
%       legend pos=outer north east,
%       legend style={draw=none},
%     ]
%     \addplot[blue,thick,domain=1:1000] {1 * x * ln(x) / ln(2)} node[near end,above] {$g(n)$};
%     \addplot[green,thick,domain=1:1000] {12 * x * ln(x) / ln(2)} node[near end,above left] (g) {$12g(n)$};
%     \addplot[orange,thick,domain=1:1000] {5 * x * ln(x) / ln(2) + 21000} node[near end,above] {$f(n)$};
%     \addplot+[red,very thick,dashed,mark=o,const plot,samples at=354] {12 * x * ln(x) / ln(2)};
%     \addplot[red,thick,dashed,const plot] coordinates {(354,0) (354,35970)};
%     %\addlegendentry{$g(n)$}; %{$n \lg(n)$}
%     %\addlegendentry{$12g(n)$}; %{$12n \lg(n)$}
%     %\addlegendentry{$f(n)$};
%     \draw[black!70,very thin,solid,text=black] (axis cs:354,35970) -- (axis cs:250,50000) node[above] {$n_0\approx 354$};
%     \draw[black!70,very thin,solid,text=black] (g.west) -> (axis cs:400,90000) node[below] {$c=12$};
%   \end{axis}
%   \end{tikzpicture}
%   \end{center}
% \end{figure}

% \begin{figure}[!htb]
%   \caption{Outro gráfico feito em \LaTeX.}
%   \label{fig:grafico2}
%   \begin{center}
%   \begin{tikzpicture}
%   \begin{axis}[
%       width=.8\textwidth,
%       ymin=0,xmin=0,xmax=150,
%       xlabel=$n$,ylabel=$T(n)$,
%       ylabel near ticks,
%       scaled ticks=false, % Evita o uso de notação exponencial 10^2
%       yticklabel style={/pgf/number format/.cd,fixed,use comma,1000 sep={}}, % Para vírgula como separador decimal
%       legend pos=north west,
%       legend style={draw=none},
%     ]
%     \addplot[blue,mark=*,thick,domain=1:150] {6 * x * ln(x) / ln(2) + 6 * x};
%     \addplot[orange,mark=square,thick,domain=1:150] {1 / 2 * x^2};
%     \addlegendentry{$6n \lg(n) + 6n$}
%     \addlegendentry{$\frac{1}{2}n^2$}
%   \end{axis}
%   \end{tikzpicture}
%   \end{center}
% \end{figure}


% \begin{figure}[!htb]
%   \caption{Variação dos resultados utilizando seleção por Janela Deslizante.}
%   \label{fig:grafico_dados}
%   \begin{center}
%   \begin{tikzpicture}[scale=1]
%     \begin{axis}[
%       width=0.9\textwidth,%height=0.6\textwidth,
%       xmode=normal,ymode=normal,
%       ymin=20,
%       xtick=data,%ticks=both,
%       xlabel=Tamanho,
%       ylabel=Acerto (\%),
%       legend pos=south east,
%       %legend style={draw=none},
%     ]
%     \addplot+[thick] table [x=tamanho,y=metodo1,header=true] {resultados.txt};
%     \addlegendentry{Método 1}
%     \addplot+[thick,mark=square] table [x=tamanho,y=metodo2,header=true] {resultados.txt};
%     \addlegendentry{Método 2}
%     \addplot+[thick,mark=triangle] table [x=tamanho,y=metodo3,header=true] {resultados.txt};
%     \addlegendentry{Método 3}
%   \end{axis}
%   \end{tikzpicture}
% \end{center}
% \end{figure}


% Figuras também podem ser incorporadas de arquivos externos, como é o caso da \Cref{fig:grafico_excel}.
%   Se a figura que ser incluída se tratar de um diagrama, um gráfico ou uma ilustração que você mesmo produza, priorize o uso de imagens vetoriais no formato PDF.
%   Com isso, o tamanho do arquivo final do trabalho será menor, e as imagens terão uma apresentação melhor, principalmente quando impressas, uma vez que imagens vetorias são perfeitamente escaláveis para qualquer dimensão.
%   Nesse caso, se for utilizar o Microsoft Excel para produzir gráficos, ou o Microsoft Word para produzir ilustrações, exporte-os como PDF e os incorpore ao documento conforme o exemplo abaixo.
%   No entanto, para manter a coerência no uso de software livre (já que você está usando \LaTeX e \abnTeX), teste a ferramenta \textsf{InkScape}\index{InkScape}\footnote{\url{http://inkscape.org/}}.
%   Ela é uma excelente opção de código-livre para produzir ilustrações vetoriais, similar ao CorelDraw\index{CorelDraw} ou ao Adobe Illustrator\index{Adobe Illustrator}.

% De todo modo, caso não seja possível utilizar arquivos de imagens como PDF, utilize qualquer outro formato, como PNG, JPEG, etc.
%   Nesse caso, você pode tentar aprimorar as imagens incorporadas com o software livre \textsf{Gimp}\index{Gimp}\footnote{\url{http://www.gimp.org/}}.
%   Ele é uma alternativa livre ao Adobe Photoshop\index{Adobe Photoshop}.

% \begin{figure}[htb]
%   \caption{Gráfico produzido em Excel e salvo como PDF.}\label{fig:grafico_excel}
%   \begin{center}
%       \includegraphics[scale=0.5]{img/abntex2-modelo-img-grafico}
%   \end{center}
%   \fonte{\citeonline[p. 24]{araujo2012}}
% \end{figure}


% A \Cref{fig:exemplo} na página \pageref{fig:exemplo} contém duas subfiguras, \Cref{subfig:exemplo:arquivo} e \subcaptionref{subfig:exemplo:tikz}.
%   A \Cref{subfig:exemplo:arquivo} foi inserida de um arquivo externo, enquanto a \Cref{subfig:exemplo:tikz} foi escrita dentro do próprio código \TeX.
%   A \Cref{fig:exemplo2} contém o mesmo exemplo, mas usando comandos diferentes para inserir as Subfiguras \ref{subfig:exemplo2:arquivo} e \subcaptionref{subfig:exemplo2:tikz}.

% \begin{figure}[!htb]
%   \centering
%   \caption{Exemplo de subfiguras.}\label{fig:exemplo}
%   \subbottom[Uma figura de um arquivo.]{\includegraphics[scale=1]{img/exemplo}\label{subfig:exemplo:arquivo}}\fonte{\citeonline{Moro2012}}
%   \qquad
%   \subbottom[Uma figura em puro código TikZ.]{%
%     \begin{tikzpicture}
%       [tipo1/.style={rectangle,draw,minimum height=13mm,minimum width=9mm},
%       tipo2/.style={circle,draw,minimum height=6mm,minimum width=6mm,
%                      prefix after command={\pgfextra{\tikzset{every label/.style={font=\footnotesize}}}}},
%       tiposeta/.style={->,shorten >=6pt,shorten <=6pt,>=triangle 60,thick}]
%       \node [tipo1] (base) [label=below:Base Station] {};
%       \node [tipo1] (sink) [right=22mm of base,label=below:Nodo Sink] {};
%       \node [tipo2] (sensor1) [above right=4mm and 17mm of sink,label=right:Sensor] {};
%       \node [tipo2] (sensor2) [right=22mm of sink,label=right:Sensor] {};
%       \node [tipo2] (sensor3) [below right=4mm and 17mm of sink,label=right:Sensor] {};
%       \draw [tiposeta] (base.east) -- (sink.west);
%       \draw [tiposeta] (sink.east) -- (sensor1.south west);
%       \draw [tiposeta] (sink.east) -- (sensor2.west);
%       \draw [tiposeta] (sink.east) -- (sensor3.north west);
%     \end{tikzpicture}
%     \label{subfig:exemplo:tikz}%
%   }
%   \legend{Alterado: de \citeonline{Moro2012}}%
% \end{figure}

% Na \Cref{fig:exemplo}, as legendas (que indicam a fonte) para cada subfigura só funcionaram porque as figuras ficaram uma embaixo da outra.
%   Se elas estivessem lado a lado, a inserção do comando \verb|\legend| em cada uma faria com que elas ficassem organizadas na vertical.
%   Uma legenda geral funcionaria, entretanto.

% Na \Cref{fig:exemplo2}, tanto legendas para subfiguras quanto uma legenda geral funcionam.

% \begin{figure}[!htb]
%   \centering
%   \caption{Mesmo exemplo de subfiguras, agora em escala.}\label{fig:exemplo2}
%   \begin{minipage}{0.48\textwidth}
%     \centering
%     \includegraphics[scale=.5]{img/exemplo}
%     \subcaption{Uma figura de um arquivo.\label{subfig:exemplo2:arquivo}}
%     \fonte{\citeonline{Moro2012}}%
%   \end{minipage}
%   %\quad
%   \begin{minipage}{0.48\textwidth}
%     \centering
%     \resizebox{0.6\textwidth}{!}{%
%       \begin{tikzpicture}[%
%          tipo1/.style={rectangle,draw,minimum height=13mm,minimum width=9mm},
%          tipo2/.style={circle,draw,minimum height=6mm,minimum width=6mm,
%                       prefix after command={\pgfextra{\tikzset{every label/.style={font=\footnotesize}}}}},
%          tiposeta/.style={->,shorten >=6pt,shorten <=6pt,>=triangle 60,thick}]
%         \node [tipo1] (base) [label=below:Base Station] {};
%         \node [tipo1] (sink) [right=22mm of base,label=below:Nodo Sink] {};
%         \node [tipo2] (sensor1) [above right=4mm and 17mm of sink,label=right:Sensor] {};
%         \node [tipo2] (sensor2) [right=22mm of sink,label=right:Sensor] {};
%         \node [tipo2] (sensor3) [below right=4mm and 17mm of sink,label=right:Sensor] {};
%         \draw [tiposeta] (base.east) -- (sink.west);
%         \draw [tiposeta] (sink.east) -- (sensor1.south west);
%         \draw [tiposeta] (sink.east) -- (sensor2.west);
%         \draw [tiposeta] (sink.east) -- (sensor3.north west);
%       \end{tikzpicture}
%     }
%     \subcaption{\label{subfig:exemplo2:tikz}Uma figura em puro código TikZ.}
%     \fonte{Alterado de \citeonline{Moro2012}}%
%   \end{minipage}
%   \fonte{Fonte geral}%
% \end{figure}


% %------------------------------------------------------------------------------
% \subsection{Sobre a indicação da fonte de uma tabela ou figura}
% %------------------------------------------------------------------------------

% As normas \citeonline[5.8]{NBR14724:2011} e o Manual de Normatização da UNIPAMPA \cite{SisbiUnipampa2011} dizem para, ``Após a ilustração, na parte inferior, indicar a fonte consultada (elemento obrigatório, mesmo que seja produção do próprio autor), legenda, notas e outras informações necessárias à sua compreensão (se houver).''
%   A primeira interpretação é a de que, mesmo que o autor tenha criado a figura, a fonte deverá ser indicada.
%   Com efeito, várias outras normas, manuais e inclusive o exemplo do pacote \abnTeX2 usam ``Fonte: os autores'' em alguns lugares.

% Entretanto, isso não está correto.
%   Veja o trecho em destaque: ``Após a ilustração, na parte inferior, indicar a fonte \textbf{consultada} (elemento obrigatório, mesmo que seja produção do próprio autor) (...).''
%   A interpretação correta é a de que, caso a ilustração tenha sido \textbf{extraída} de um documento, a fonte deve ser indicada, ainda que esse documento pertença ao próprio autor.
%   A sentença original das normas deveria ter sido melhor escrita para evitar a interpretação incorreta.

% Assim, não indique a fonte se a figura ou tabela for original, ou seja, foi criada para o trabalho.
%   Caso contrário, indique a fonte.
%   Mas cuidado: caso a figura ou tabela tenha sido adaptada de outra já publicada, então é obrigatório indicar ``adaptado de'' ou ``acrescida de'' seguido da referência da fonte de onde ela foi extraída.


% %------------------------------------------------------------------------------
% \section{Expressões matemáticas}
% %------------------------------------------------------------------------------

% \index{expressões matemáticas}Use o ambiente \texttt{equation} para escrever expressões matemáticas numeradas:

% \begin{equation}
%   \forall x \in X, \quad \exists \: y \leq \epsilon
% \end{equation}

% Escreva expressões matemáticas entre \$ e \$, como em $\lim_{x \to \infty} \exp(-x) = 0$, para que fiquem na mesma linha.

% Também é possível usar colchetes para indicar o início de uma expressão matemática que não é numerada.

% \[
% \left|\sum_{i=1}^n a_ib_i\right|
% \le
% \left(\sum_{i=1}^n a_i^2\right)^{1/2}
% \left(\sum_{i=1}^n b_i^2\right)^{1/2}
% \]

% Consulte mais informações sobre expressões matemáticas em \url{http://code.google.com/p/abntex2/w/edit/Referencias}.


% %------------------------------------------------------------------------------
% \section{Enumerações: alíneas e subalíneas}
% %------------------------------------------------------------------------------

% \index{alíneas}\index{subalíneas}\index{incisos}Quando for necessário enumerar os diversos assuntos de uma seção que não possua título, esta deve ser subdividida em alíneas \cite[4.2]{NBR6024:2012}:

% \begin{alineas}

%   \item os diversos assuntos que não possuam título próprio, dentro de uma mesma
%   seção, devem ser subdivididos em alíneas\footnote{As notas devem ser digitadas ou datilografadas
%   dentro das margens, ficando separadas do texto por um espaço simples de entre as
%   linhas e por filete de 5 cm, a partir da margem esquerda. Devem ser
%   alinhadas, a partir da segunda linha da mesma nota, abaixo da primeira letra
%   da primeira palavra, de forma a destacar o expoente, sem espaço entre elas e
%   com fonte menor. \citeonline[5.2.1]{NBR14724:2011}};

%   \item o texto que antecede as alíneas termina em dois pontos;
%   \item as alíneas devem ser indicadas alfabeticamente, em letra minúscula, seguida de parêntese. Utilizam-se letras dobradas, quando esgotadas as letras do alfabeto;

%   \item as letras indicativas das alíneas devem apresentar recuo em relação à
%   margem esquerda;

%   \item o texto da alínea deve começar por letra minúscula e terminar em
%   ponto-e-vírgula, exceto a última alínea que termina em ponto final;

%   \item o texto da alínea deve terminar em dois pontos, se houver subalínea;

%   \item a segunda e as seguintes linhas do texto da alínea começa sob a
%   primeira letra do texto da própria alínea;

%   \item subalíneas \cite[4.3]{NBR6024:2012} devem ser conforme as alíneas a
%   seguir:

%   \begin{alineas}
%      \item as subalíneas devem começar por travessão seguido de espaço;

%      \item as subalíneas devem apresentar recuo em relação à alínea;

%      \item o texto da subalínea deve começar por letra minúscula e terminar em
%      ponto-e-vírgula. A última subalínea deve terminar em ponto final, se não
%      houver alínea subsequente;

%      \item a segunda e as seguintes linhas do texto da subalínea começam sob a
%      primeira letra do texto da própria subalínea.
%   \end{alineas}

%   \item no \abnTeX\ estão disponíveis os ambientes \texttt{incisos} e \texttt{subalineas}, que em suma são o mesmo que se criar outro nível de \texttt{alineas}, como nos exemplos à seguir:

%   \begin{incisos}
%     \item \textit{Um novo inciso em itálico};
%   \end{incisos}

%   \item Alínea em \textbf{negrito}:

%   \begin{subalineas}
%     \item \textit{Uma subalínea em itálico};
%     \item \underline{\textit{Uma subalínea em itálico e sublinhado}};
%   \end{subalineas}

%   \item Última alínea com \emph{ênfase}.

% \end{alineas}


% %------------------------------------------------------------------------------
% \section{Espaçamento entre parágrafos e linhas}
% %------------------------------------------------------------------------------

% \index{espaçamento!dos parágrafos}O tamanho do parágrafo, espaço entre a margem e o início da frase do parágrafo, é definido por:

% \begin{verbatim}
%   \setlength{\parindent}{1.3cm}
% \end{verbatim}

% \index{espaçamento!do primeiro parágrafo}Por padrão, não há espaçamento no primeiro parágrafo de cada início de divisão do documento (\Cref{sec:divisoes}).
%   Porém, você pode definir que o primeiro parágrafo também seja indentado, como é o caso deste documento.
%   Para isso, apenas inclua o pacote \textsf{indentfirst} no preâmbulo do documento:
% \begin{verbatim}
%   \usepackage{indentfirst}      % Indenta o primeiro parágrafo de cada seção.
% \end{verbatim}

% \index{espaçamento!entre os parágrafos}O espaçamento entre um parágrafo e outro pode ser controlado por meio do comando:
% \begin{verbatim}
%   \setlength{\parskip}{0.2cm}  % tente também \onelineskip
% \end{verbatim}

% \index{espaçamento!entre as linhas}O controle do espaçamento entre linhas é definido por:
% \begin{verbatim}
%   \OnehalfSpacing       % espaçamento um e meio (padrão);
%   \DoubleSpacing        % espaçamento duplo
%   \SingleSpacing        % espaçamento simples
% \end{verbatim}

% Para isso, também estão disponíveis os ambientes:
% \begin{verbatim}
%   \begin{SingleSpace} ...\end{SingleSpace}
%   \begin{Spacing}{hfactori} ... \end{Spacing}
%   \begin{OnehalfSpace} ... \end{OnehalfSpace}
%   \begin{OnehalfSpace*} ... \end{OnehalfSpace*}
%   \begin{DoubleSpace} ... \end{DoubleSpace}
%   \begin{DoubleSpace*} ... \end{DoubleSpace*}
% \end{verbatim}

% Para mais informações, consulte \citeonline[p. 47-52 e 135]{memoir}.


% %------------------------------------------------------------------------------
% \section{Inclução de outros arquivos}\label{sec:include}
% %------------------------------------------------------------------------------

% É uma boa prática dividir o seu documento em diversos arquivos, e não apenas escrever tudo em um único.
%   Esse recurso foi utilizado neste documento.
%   Para incluir diferentes arquivos em um arquivo principal, de modo que cada arquivo incluído fique em uma página diferente, utilize o comando:
% \begin{verbatim}
%   \include{documento-a-ser-incluido}      % sem a extensão .tex
% \end{verbatim}

% Para incluir documentos sem quebra de páginas, utilize:
% \begin{verbatim}
%   \input{documento-a-ser-incluido}      % sem a extensão .tex
% \end{verbatim}


% %------------------------------------------------------------------------------
% \section{Compilar o documento \LaTeX}
% %------------------------------------------------------------------------------

% Geralmente os editores \LaTeX, como o TeXlipse\footnote{\url{http://texlipse.sourceforge.net/}}, o Texmaker\footnote{\url{http://www.xm1math.net/texmaker/}}, entre outros, compilam os documentos automaticamente, de modo que você não precisa se preocupar com isso.

% No entanto, você pode compilar os documentos \LaTeX usando os seguintes comandos, que devem ser digitados no \emph{Prompt de Comandos} do Windows ou no \emph{Terminal} do Mac ou do Linux:
% \begin{verbatim}
%   pdflatex ARQUIVO_PRINCIPAL.tex
%   bibtex ARQUIVO_PRINCIPAL.aux
%   makeindex ARQUIVO_PRINCIPAL.idx
%   makeindex ARQUIVO_PRINCIPAL.nlo -s nomencl.ist -o ARQUIVO_PRINCIPAL.nls
%   pdflatex ARQUIVO_PRINCIPAL.tex
%   pdflatex ARQUIVO_PRINCIPAL.tex
% \end{verbatim}


\section{Embasamento}
\subsection*{Embasamento}

\begin{frame}{{\sffamily Curricularização da Extensão}}
  \begin{block}{}
    \begin{itemize}
      \item Política Nacional de Extensão
            \SubItem{Extensão como solução para problemas sociais}
            \SubItem{Extensão como ferramenta essencial}
            \SubItem{Promover solidariedade e conscientização ambiental}
      \item Parte do curso como extensão (Res. 7 de 2018)
            \SubItem{10\% de horas do total}
            \SubItem{Adaptação em até 3 anos}
    \end{itemize}

  \end{block}
\end{frame}

\begin{frame}{{\sffamily Curricularização da Extensão na UNIPAMPA}}
  \begin{block}{}
    \begin{itemize}
      \item Disciplina de Resolução de Problemas
      \item Objetivos da Extensão pela (Res. 317 de 2021)
            \SubItem{Desenvolver a educação crítica, cívica, interdisciplinar e responsável}
            \SubItem{Fortalecer o vínculo entre ensino, pesquisa e extensão}
      \item Formalização pela (Res. 332 de 2021)
            \SubItem{Órgãos gestores, participantes da extensão, tramitações}% Quem coordena, quem participa, documentos necessarios, autoavaliação, minimo de duraç~ao de 8 horas
    \end{itemize}

  \end{block}
\end{frame}

\begin{frame}{{\sffamily Programas e Projetos de Extensão}}
  % Objetivo não é criar um laço de dependência entre comunidades, so para resolver um determinado problema e deixar que eles consigam andar com as proprias pernas depois, por isso a qualidade
  \begin{block}{O que são}
    Ações que envolvem ensino, pesquisa e a comunidade externa.
    \begin{itemize}
      \item Projetos possuem um objetivo específico e prazos determinados % Da pra falar das roles
      \item Um programa é um conjunto de projetos
    \end{itemize}
  \end{block}

  \begin{block}{Exemplos de programas}
    \begin{itemize}
      \item Programa C % Professora Aline
      \item Programa JEDI % Professor Bernardino
      \item Programa TRAMAS % Professor Diego
    \end{itemize}
  \end{block}

\end{frame}

\begin{frame}{{\sffamily Unipampa Cidadã}}
  \begin{block}{Unipampa Cidadã}
    \begin{itemize}
      \item Extensão curricular
      \item Atividades solidárias % Campanha do agasalho, 
            \SubItem{Campanha do agasalho, arrecadação de alimentos, suporte a asilos}
      \item Formação de egressos socialmente responsáveis % Egressos mais consientes com a responsabilidade social, Integração com a comunidade local
      \item Oferecida por todos os cursos
            \SubItem{Mínimo de 60 e máximo de 120 horas}
      \item Formulário de finalização da atividade
    \end{itemize}
  \end{block}
\end{frame}

%######################################################################
% \begin{frame}{{\sffamily  Revisão Sistemática}}
%     \begin{figure}
%         \centering
%         \caption{Processo de Revisão Sistemática de Literatura (NAKAGAWA, E. et al.)} 
%         \includegraphics[width=0.68\textwidth]{Img/ProtocoloRS2.jpg}
%     \end{figure}
% \end{frame}
%######################################################################

% \begin{frame}{{\sffamily  Revisão Sistemática: Critérios de Inclusão}}
% \begin{block}{}
% 	\begin{itemize}%[<+->]
% 	        \item{CI1.O estudo deve citar uma ferramenta de teste de desempenho;}
% 	        \item{CI2.O estudo deve possuir avaliação empírica e expositiva sobre a ferramenta abordada;}
% 	        \item{CI3.Métricas e/ou padrões/normas de qualidade que apoiam a avaliação de uma ferramenta de teste de desempenho;}
% 	\end{itemize}
% \end{block}
% \end{frame}

% \begin{frame}{{\sffamily  Revisão Sistemática: Critérios de Exclusão}}
% \begin{block}{}
% 	\begin{itemize}%[<+->]
% 	        \item{CE1.Estudos duplicados;}
% 	        \item{CE2.O estudo não está escrito em inglês;}
% 	        \item{CE3.O estudo não fornece acesso completo ao seu conteúdo;}
% 	        \item{CE4.O estudo possui menos de 6 páginas, i.e., resumos expandidos;}
% 	\end{itemize}
% \end{block}
% \end{frame}

% \begin{frame}{{\sffamily  Revisão Sistemática: Goal Questions and Metrics}}
%     \begin{figure}
%         \centering
%         \includegraphics[width=0.95\textwidth]{Img/gqm.png}
%     \end{figure}
% \end{frame}

% \begin{frame}{{\sffamily  Revisão Sistemática: Questões de Pesquisa}}
% \begin{block}{}
% 	\begin{itemize}%[<+->]
% 	        \item{G1.Identificar quais são as ferramentas de teste de desempenho, e mapear suas funcionalidades.}
% 	        \SubItem{QP1.Quais são as ferramentas que apoiam o teste de desempenho?}
% 	        \SubItem{QP2.Quais as funcionalidades das ferramentas de teste de desempenho?}
% 	        \SubItem{QP3.Quais as abordagens de integração entre os artefatos de teste gerados nas ferra-mentas de teste de desempenho?}
% 	 \end{itemize}
% \end{block}
% \end{frame}

% \begin{frame}{{\sffamily  Revisão Sistemática: Questões de Pesquisa}}
% \begin{block}{}
% 	\begin{itemize}%[<+->]
% 	        \item{G2.Identificar abordagens, bibliotecas e ferramentas para visualização de dados,as quais possam ser utilizadas na análise de dados em relatórios de teste de desempenho.}
% 	        \SubItem{QP4.Quais as abordagens de apresentação dos resultados do teste de desempenho?}
% 	        \SubItem{QP5.Quais as abordagens de análise de resultados do teste de desempenho?}
% 	        \SubItem{QP6.Quais são as bibliotecas e ferramentas de visualização de dados a partir de uma entrada de dados?}
% 	 \end{itemize}
% \end{block}
% \end{frame}

% \begin{frame}{{\sffamily  Revisão Sistemática: Métricas}}
% \begin{block}{}
% 	\begin{itemize}%[<+->]
% 	        \item{M1.Nomes das ferramentas de teste de desempenho encontradas no estudo;}
% 	        \item{M2.Quais as funcionalidades das ferramentas;}
% 	        \item{M3.Abordagem de apresentação dos dados;}
% 	        \item{M4.Bibliotecas e ferramentas de visualização de dados.}
%     \end{itemize}
% \end{block}
% \end{frame}


% \begin{frame}{{\sffamily  Revisão Sistemática: String de Busca}}

% \begin{table}[!h]
% \centering
% \footnotesize
% \label{tab:DefinicaoSearch}
% \begin{tabular}{m{3cm}|m{6cm}}
% \bottomrule
% \rowcolor[HTML]{C0C0C0}
% \textbf{Termos} & \textbf{Sinônimos}
%  \\
% %\midrule 
% \rowcolor[HTML]{EFEFEF}
% \hline
% {\textbf{Performance Test}} & Load test, Stress test, Spike test, Workload test, Automation test 
%  \\
% %\midrule
% \hline
% \textbf{Tool} & Monitor, Generator, Plugin, Plug-in, Framework, Injector, Suite, Analyzer, Environment
% \\
% %\midrule
% \hline
% \toprule
% \end{tabular}
% \end{table}


% \begin{figure}[htb]
% \centering
% \fbox{\
% \normalsize
% \parbox{10cm}{
% \centering
% \scriptsize\texttt{
% (Performance test OR Load test OR Stress test OR Spike test OR Soak test OR Workload test OR Automation test) AND (Tool OR Monitor OR Generator OR Plugin OR Plug-in OR Framework OR Injector OR Suite OR Environment) AND (Software OR Application OR System)}}}
% \label{fig:StringBusca}
% \end{figure}

% \end{frame}

% \begin{frame}{{\sffamily  Revisão Sistemática: Estudos Primários}}
%     \begin{figure}[!h]
% 	\centering
% 	\caption{Estudos primários retornados nas bases de dados} 
% 		\includegraphics[width=.70\textwidth]{Img/ResultadosBases.JPG}
% 	\label{fig:ResusltadosBases}
% \end{figure}
%  \end{frame}

% % \begin{frame}{{\sffamily  Revisão Sistemática: Resultados dos Estudos Primários}}
% %     \begin{figure}
% %         \centering
% %         \includegraphics[width=0.99\textwidth]{Img/ResultadosBases.JPG}
% %         \end{figure}
% % \end{frame}

% \begin{frame}{{\sffamily  Revisão Sistemática: Processo de Seleção dos Estudos. }}
% \begin{figure}
%     \centering
%     \caption{Processo de Seleção dos Estudos.}
%     \includegraphics[width=0.70\textwidth]{Img/ciclos.PNG}
% \end{figure}
% \end{frame}

% % \begin{frame}{{\sffamily  Revisão Sistemática: Critérios de Qualidade}}
% % \begin{block}{}
% % 	\begin{itemize}%[<+->]
% %         \item{CQ1.Existem quaisquer declarações dos objetivos de pesquisa no estudo analisado?}
% %         \item{CQ2.O estudo apresenta características da ferramenta?}
% %         \item{CQ3.A proposta central do estudo está definida e é fortemente ligada às questões de pesquisa?}
% %         \item{CQ4.O estudo exibe as características e os recursos da análise dos resultados do teste de desempenho?}
% %         \item{CQ5.O estudo exibe detalhes de como é feita a integração dos artefatos gerados após o teste, para a análise dos resultados do teste de desempenho?}
% % 	\end{itemize}
% % \end{block}
% % \end{frame}

% \begin{frame}{{\sffamily  Revisão Sistemática: Avaliação de Qualidade}}
% \begin{figure}
%     \centering
%     \includegraphics[width=0.90\textwidth]{Img/qualidade.PNG}
% \end{figure}
% \end{frame}

% \begin{frame}{{\sffamily  Revisão Sistemática: Ferramentas Citadas nos Estudos}}
% \begin{figure}
%     \centering
%     \includegraphics[width=0.45\textwidth]{Img/ferramentas.PNG}
% \end{figure}
% \end{frame}

% \begin{frame}{{\sffamily  Revisão Sistemática: Funcionalidades das Ferramentas}}
% \begin{figure}
%     \centering
%     \includegraphics[width=0.52\textwidth]{Img/funcionalidades.PNG}
% \end{figure}
% \end{frame}

% \begin{frame}{{\sffamily  Revisão Sistemática: Formatos e Características do Relatório}}
% \begin{figure}
%     \centering
%     \includegraphics[width=0.35\textwidth]{Img/graficosferramentas.PNG}
% \end{figure}
% \end{frame}

% \begin{frame}{{\sffamily  Revisão Sistemática: Distribuição Geográfica dos Estudos}}
% \begin{figure}
%     \centering
%     \caption{Distribuição Geográfica dos Estudos} 
%     \includegraphics[width=0.95\textwidth]{Img/mapa.PNG}
% \end{figure}
% \end{frame}

% % \begin{frame}{{\sffamily  Revisão Sistemática: Anos de Publicação dos Estudos Selecionados.}}
% %     \begin{figure}
% %         \centering
% %         \includegraphics[width=0.70\textwidth]{Img/anosdepublicacao.PNG}
% %         \end{figure}
% % \end{frame}

% \begin{frame}{{\sffamily  Busca na Literatura Cinza: Motivação}}
%         \begin{block}{}
% 	\begin{itemize}%[<+->]
% 	    \item  Necessidade de encontrar ferramentas e bibliotecas de visualização de dados.
% 	    \item  \textit{Guidelines for including \textbf{Grey Literature} and conducting multivocal literature reviews in software engineering}. GAROUSI, V.
% 	    \end{itemize}
% \end{block}
% \end{frame}

% \begin{frame}{{\sffamily  Busca na Literatura Cinza: Ferramentas e Bibliotecas de Visualização de Dados}}
%     \begin{figure}
%         \centering
%         \includegraphics[width=0.33\textwidth]{Img/bibliotecas.png}
%         \end{figure}
% \end{frame}

% %######################################################################
% \section{Proposta}
% \subsection*{Proposta}
% %######################################################################


% \begin{frame}{{\sffamily  Requisitos}}
%     \begin{block}{}
% 	\begin{itemize}%[<+->]
% 	    \item  RQ1.A ferramenta precisa estar disponível sob licença open source.
% 	    \item  RQ2.A ferramenta tem de ser capaz de receber uma entrada de dados como parâmetro.
% 	    \item  RQ3.A ferramenta deve ser capaz de gerar gráficos a partir da entrada de dados.
% 	    \item  RQ4.A ferramenta deve ser capaz de apresentar dados de forma interativa.
% 	    \item  RQ5.A ferramenta deve automatizar a análise de resultados.
% 	\end{itemize}
% \end{block}
% \end{frame}

% \begin{frame}{{\sffamily Decisões de Projeto}}
%     \begin{block}{}
% 	\begin{itemize}%[<+->]
% 	    \item DP1. A ferramenta/biblioteca de visualização de dados deve ser \textit{open-source} (RQ1).
% 	    \item DP2. A ferramenta/biblioteca de visualização de dados deve ser capaz de receber uma entrada de dados (RQ2).
% 	    \item DP3. A ferramenta/biblioteca de visualização de dados tem de gerar gráficos a partir da entrada de dados (RQ3).
% 	\end{itemize}
% \end{block}
% \end{frame}

% \begin{frame}{{\sffamily Decisões de Projeto}}
%     \begin{block}{}
% 	\begin{itemize}%[<+->]
% 	    \item DP4. A ferramenta/biblioteca de visualização de dados deve possibilitar a apresentação de dados de forma interativa (RQ4).
% 	    \item DP5. A ferramenta deve fornecer alternativas para a análise de resultados automatizada (RQ5).
% 	\end{itemize}
% \end{block}
% \end{frame}

% \begin{frame}{{\sffamily  Arquitetura}}
%     \begin{figure}
%         \centering
%         \caption{Arquitetura da Proposta} 
%         \includegraphics[width=0.85\textwidth]{Img/arquiteturaearth.png}
%         \end{figure}
% \end{frame}

% \begin{frame}{{\sffamily  Diagrama de Caso de Uso}}
%     \begin{figure}
%         \centering
%         \includegraphics[width=0.75\textwidth]{Img/casodeuso.png}
%         \end{figure}
% \end{frame}

% \begin{frame}{{\sffamily  Caso de Uso 1. A ferramenta deve disponibilizar métricas \\para a análise dos resultados.}}
% \begin{figure}
%     \centering
%     \includegraphics[width=0.33\textwidth]{Img/earthresultados.PNG}
% \end{figure}
% \end{frame}

% \begin{frame}{{\sffamily  Caso de Uso 2. A ferramenta deve ser capaz de gerar \\gráficos a partir da entrada de dados}}
% \begin{figure}
%     \centering
%     \includegraphics[width=0.95\textwidth]{Img/earthnalysis.png}
% \end{figure}
% \end{frame}

% %######################################################################
% \section{Avaliação}
% \subsection*{Mapeamento das Necessidades}

% %######################################################################

% \begin{frame}{{\sffamily  Mapeamento das Necessidades}}
%     \begin{block}{}
% 	\begin{itemize}%[<+->]
% 	    \item  Planejamento;
% 	    \SubItem{Estrutura estabelecida}
% 	    \SubItem{Estratégia das questões}
% 	    \item  Objetivo;
% 	    \SubItem{Entendimento das necessidades da indústria}
% 	    \SubItem{Coletar informações de profissionais da área}
% 	    \item  Execução.
% 	    \SubItem{Distribuição do questionário}
% 	    \SubItem{52 participantes, 8 países*}
% 	\end{itemize}
% \end{block}
% \end{frame}
% %######################################################################

% %######################################################################

% \begin{frame}{{\sffamily  Mapeamento das Necessidades - Perfil dos Respondentes}}
% \begin{block}{}
% \begin{figure}
%     \centering
%     \includegraphics[width=0.25\textwidth]{Img/empresasmapeamento.png}
% \end{figure}
% \end{block}
% \end{frame}
% %######################################################################

% %######################################################################

% \begin{frame}{{\sffamily  Mapeamento das Necessidades - Nível Acadêmico}}
% \begin{block}{}
% \begin{figure}
%     \centering
%     \caption{Nível de formação acadêmica dos respondentes}
%     \includegraphics[width=0.85\textwidth]{Img/educacaomapeamento.png}
% \end{figure}
% \end{block}
% \end{frame}
% %######################################################################

% %######################################################################

% \begin{frame}{{\sffamily  Mapeamento das Necessidades - Nível de Experiência}}
% \begin{block}{}
%     \begin{figure}
%         \centering
%         \caption{Anos de Experiência na Indústria}
%         \includegraphics[width=0.85\textwidth]{Img/experienciamapeamento.png}
%     \end{figure}
%     \begin{figure}
%         \centering
%          \caption{Anos de Experiência como Engenheiro de Teste de Desempenho}
%         \includegraphics[width=0.85\textwidth]{Img/experienciatestermapeamento.png}
%     \end{figure}
% \end{block}
% \end{frame}
% %######################################################################

% %######################################################################

% \begin{frame}{{\sffamily  Mapeamento das Necessidades - Resultados}}
% \begin{block}{}
% 	\begin{itemize}%[<+->]
% 	    \item  Q1.\textit{How do you analyze the results of the performance test?} Do you use any tool? Which? (Como você analisa os resultados do teste de desempenho? Você usa alguma ferramenta? Qual?);
%             \SubItem{\textit{JMeter}, \textit{Dynatrace} e \textit{LoadRunner.}}
% 	    \item  Q2. \textit{What feature could help the process of analyzing the results?} (Que recurso poderia ajudar no processo de análise dos resultados?);
% 	        \SubItem{Gráficos, Comparação entre os testes executados, Customização, Visualização do tempo de resposta frente a outras métricas.}
% 	\end{itemize}
% \end{block}
% \end{frame}
% %######################################################################

% %######################################################################

% \begin{frame}{{\sffamily  Mapeamento das Necessidades - Questões Abertas}}
%     \begin{block}{}
% 	\begin{itemize}%[<+->]
% 	     \item  Q3. \textit{How do you rate the resources for analyzing results in performance testing tools?} (Como você classifica os recursos para análise de resultados em ferramentas de teste de desempenho?)
% 	     \item  Q4. \textit{How beneficial would be a solution that could be coupled with a performance test project, to have resources for the analysis of results?} (Quão benéfica seria uma solução que pudesse ser acoplada a um projeto de teste de desempenho, para ter recursos para a análise de resultados?)
% 	\end{itemize}
% \end{block}
% \end{frame}
% %######################################################################

% \begin{frame}{{\sffamily  Mapeamento das Necessidades - Questões Fechadas}}
% \begin{block}{}
% \begin{figure}
%     \centering
%     \caption{Classificação de 1 - 10 para pergunta \textbf{Q3.} e \textbf{Q4.}}
%     \includegraphics[width=0.95\textwidth]{Img/questoes.png}
% \end{figure}
% \end{block}
% \end{frame}
% %######################################################################

% \begin{frame}{{\sffamily  Mapeamento das Necessidades - Respostas}}
% \begin{block}{}
% 	\begin{itemize}%[<+->]
% 	    \item  Q5. \textit{Do you consider that generating graphs of the results is beneficial for the analysis of the results?} (Você considera que a geração de gráficos dos resultados é benéfica para a análise dos resultados?)
% 	\end{itemize}
% 	\begin{figure}
%         \centering
%         \includegraphics[width=0.95\textwidth]{Img/benefico.png}
%     \end{figure}
% \end{block}
% \end{frame}
% %######################################################################

% %######################################################################
% %\section{Avaliação}
% \subsection*{Avaliação Exploratória}

% %######################################################################

% \begin{frame}{{\sffamily Avaliação Exploratória}}
%     \begin{block}{}
% 	\begin{itemize}%[<+->]
% 	    \item  Planejamento;
% 	    \SubItem{Estrutura estabelecida}
% 	    \SubItem{Estratégia das questões}
% 	    \item  Objetivo;
% 	    \SubItem{Entendimento das necessidades da indústria}
% 	    \SubItem{Coletar informações de profissionais da área}
% 	    \item  Execução.
% 	    \SubItem{Distribuição do questionário}
% 	    \SubItem{15 participantes, 5 países*}
% 	\end{itemize}
% \end{block}
% \end{frame}
% %######################################################################

% %######################################################################

% \begin{frame}{{\sffamily Avaliação Exploratória - Perfil dos Respondentes}}
% \begin{block}{}
% \begin{figure}
%     \centering
%     \includegraphics[width=0.60\textwidth]{Img/empresasavaliacao.png}
% \end{figure}
% \end{block}
% \end{frame}
% %######################################################################

% \begin{frame}{{\sffamily  Avaliação Exploratória - Nível acadêmico}}
% \begin{block}{}
% \begin{figure}
%     \centering
%     \caption{Nível de formação acadêmica dos respondentes}
%     \includegraphics[width=0.85\textwidth]{Img/educacaoavaliacao.png}
% \end{figure}
% \end{block}
% \end{frame}
% %######################################################################

% \begin{frame}{{\sffamily  Avaliação Exploratória - Nível de Experiência}}
% \begin{block}{}
%     \begin{figure}
%         \centering
%         \caption{Anos de Experiência na Indústria}
%         \includegraphics[width=0.85\textwidth]{Img/experienciaavaliacao.png}
%     \end{figure}
%     \begin{figure}
%         \centering
%         \caption{Anos de Experiência como Engenheiro de Teste de Desempenho}
%         \includegraphics[width=0.85\textwidth]{Img/experienciatesteravaliacao.png}
%   \end{figure}
% \end{block}
% \end{frame}
% %######################################################################

% \begin{frame}{{\sffamily Avaliação Exploratória - Avaliação Técnica}}
%     \begin{block}{}
% 	\begin{itemize}%[<+->]
% 	    \item \textbf{QF1.} \textit{The tool help in analyzing performance test results.} (A ferramenta ajuda na análise dos resultados dos testes de desempenho.)
%         \item \textbf{QF2.} \textit{I would recommend this tool to a friend.} (Eu recomendaria esta ferramenta a um amigo.)
%         \item \textbf{QF3.} \textit{Using the tool can reduce the time spent analyzing the performance test results?} (O uso da ferramenta pode reduzir o tempo gasto analisando os resultados dos testes de desempenho?)
% 	\end{itemize}
% \end{block}
% \end{frame}
% %######################################################################

% \begin{frame}{{\sffamily Avaliação Exploratória - Avaliação Técnica}}
%     \begin{block}{}
% 	\begin{itemize}%[<+->]
%         \item \textbf{QF4.} \textit{Earthnalysis produce the expected graphs for the analysis of performance testing results.} (Earthnalysis produz os gráficos esperados para a análise dos resultados dos testes de desempenho.)
%         \item \textbf{QF5.} \textit{Using the tool can improve my performance when analyzing results of performance tests.} (O uso da ferramenta pode melhorar meu desempenho ao analisar resultados de testes de desempenho.)
%         \item \textbf{QF6.} \textit{The tool is fast in generating graphics and processing .CSV.} (A ferramenta é rápida na geração de gráficos e no processamento de .CSV.)
% 	\end{itemize}
% \end{block}
% \end{frame}
% %######################################################################

% \begin{frame}{{\sffamily  Avaliação Exploratória - Respostas}}
% \begin{block}{}
%     \begin{figure}
%         \centering
%         \caption{Respostas da Avaliação Técnica}
%         \includegraphics[width=0.85\textwidth]{Img/avaliacaotecnica.PNG}
%     \end{figure}
% \end{block}
% \end{frame}
% %######################################################################


% \begin{frame}{{\sffamily Avaliação Exploratória - Avaliação de Usabilidade}}
%     \begin{block}{}
% 	\begin{itemize}%[<+->]
%         \item \textbf{QU1.} \textit{Earthnalysis was easy to use.} (Earthnalysis foi fácil de usar.)
%         \item \textbf{QU2.} \textit{Earthnalysis is a good idea.} (Earthnalysis é uma boa ideia.)
%         \item \textbf{QU3.} \textit{I liked using Earthnalysis.} (Eu gostei de usar a Earthnalysis.)
%         \item \textbf{QU4.} \textit{Earthnalysis can facilitate the analysis of performance test results.} (Earthnalysis pode facilitar a análise dos resultados dos testes de desempenho.)
% 	\end{itemize}
% \end{block}
% \end{frame}
% %######################################################################

% \begin{frame}{{\sffamily  Avaliação Exploratória - Respostas}}
% \begin{block}{}
%     \begin{figure}
%         \centering
%         \caption{Respostas da Avaliação de Usabilidade}
%         \includegraphics[width=0.95\textwidth]{Img/avaliacaodeusabilidade.PNG}
%     \end{figure}
% \end{block}
% \end{frame}
% %######################################################################

% \begin{frame}{{\sffamily  valiação Exploratória - Questões Abertas}}
%     \begin{block}{}
% 	\begin{itemize}%[<+->]
% 	     \item \textbf{QA1.} \textit {What were the positive points in your experience with the tool?} (Quais foram os pontos positivos em sua experiência com a ferramenta?)
% 	        \SubItem{Facilidade de Uso, Intuitiva, Rápida geração de gráficos.}
% 	     \item \textbf{QA2.} \textit {What were the negative points in your experience with the tool?} (Quais foram os pontos negativos em sua experiência com a ferramenta?)
% 	        \SubItem{Interface muito simples, apenas um formato aceito como entrada de dados.}
% 	\end{itemize}
% \end{block}
% \end{frame}
% %######################################################################

% \begin{frame}{{\sffamily Avaliação Exploratória - Avaliação de Afeição}}
%     \begin{block}{}
% 	\begin{itemize}%[<+->]
% 	    \item \textit{Modelo do Circumplexo de Russell} 
%         \item \textbf{QR1.} \textit {Mood while performing the evaluation.} (Humor ao realizar a avaliação.)
%         \item \textbf{QR2.} \textit {Mood while using the tool.} (Humor ao usar a ferramenta.)
% 	\end{itemize}
% \end{block}
% \end{frame}
% %######################################################################

% %######################################################################

% \begin{frame}{{\sffamily  Avaliação Exploratória - Níveis de Afeição }}
% \begin{block}{}
%   \begin{figure}
%         \centering
%         \caption{Opções disponíveis para responder a QR1 e QR2.}
%         \includegraphics[width=0.99\textwidth]{Img/afeicao.PNG}
%     \end{figure}
% \end{block}
% \end{frame}
% %######################################################################

% %######################################################################

% \begin{frame}{{\sffamily  Avaliação Exploratória - Respostas}}
% \begin{block}{}
%   \begin{figure}
%         \centering
%         \caption{Humor enquanto executou a avaliação} 
%         \includegraphics[width=0.60\textwidth]{Img/russell.PNG}
%     \end{figure}
% \end{block}
% \end{frame}
% %######################################################################

% %######################################################################

% \begin{frame}{{\sffamily  Avaliação Exploratória - Respostas}}
% \begin{block}{}
% \begin{figure}
%     \centering
%     \caption{Humor enquanto utilizou a ferramenta} 
%     \includegraphics[width=0.55\textwidth]{Img/russell2.PNG}
% \end{figure}
% \end{block}
% \end{frame}
% %######################################################################

% \begin{frame}{{\sffamily Avaliação Exploratória - Ameaças à Validade do Estudo}}
%     \begin{block}{}
% 	\begin{itemize}%[<+->]
% 	    \item Validade do Constructo:
% 	        \SubItem{Criação de perguntas embasadas nos trabalhos relacionados.}
%         \item Validade Interna:
%             \SubItem{Estratégias para reduzir cansaço e desmotivação.}
%         \item Validade Externa:
%             \SubItem{Amostra representativa.}
%         \item Validade de Conclusão:
%             \SubItem{Estratégias para não ter um público enviesado.}
% 	\end{itemize}
% \end{block}
% \end{frame}
% %######################################################################


% %######################################################################
% \section{Considerações Finais}
% \subsection*{Considerações Finais}

% %######################################################################

% \begin{frame}{{\sffamily  Atividades Desenvolvidas}}
%     \begin{block}{}
% 	\begin{itemize}%[<+->]
% 	    \item  Levantamento do estado da arte;
% 	    \item  Elaboração da proposta;
% 	    \item  Desenvolvimento da ferramenta;
% 	    \item  Avaliação da ferramenta.
% 	\end{itemize}
% \end{block}
% \end{frame}

% %######################################################################

% \subsection*{Considerações Finais}

% %######################################################################

% \begin{frame}{{\sffamily  Lições Aprendidas e Relatos de Experiência}}
%     \begin{block}{}
% 	\begin{itemize}%[<+->]
% 	    \item Dificuldades encontradas;
% 	    \item Teste de desempenho na indústria;
% 	    \item Adesão na utilização da ferramenta desenvolvida.
% 	\end{itemize}
% \end{block}
% \end{frame}

% %######################################################################

% \begin{frame}{{\sffamily Earthnalysis - Uma Proposta de Visualização de Dados \\a partir de Relatórios de Teste de Desempenho}}
% \begin{block}{\begin{center}\huge{Perguntas?}\end{center}}
% 	\begin{center}
% 		\huge{}
% 	\end{center}
% \end{block}
% \begin{block}{Sobre:}
% 	\begin{itemize}%[<+->]
% 	    \item \href{http://lesse.com.br/}{{\color{blue}lesse.com.br}}
% 	    \item \href{http://novoportal.unipampa.edu.br/novoportal/}{{\color{blue}unipampa.edu.br}}
%     \end{itemize}
% \end{block}
% \end{frame}

% %######################################################################

\begin{frame}[plain,t]
  \titlepage
\end{frame}

\end{document}
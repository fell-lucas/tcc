\begin{resumo}
  Em 2023, o processo de curricularização de novas Ações de Extensão será implantado obrigatoriamente pelas Instituições de Ensino Superior do país. Apesar disso, em sua maioria, as Instituições não possuem um processo completamente automatizado para a gestão dos Programas e Projetos de Extensão, que continuaria sendo realizada manualmente pelo coordenador ou colaboradores de extensão. A realidade não é diferente na \acs{UNIPAMPA}, onde foi inicialmente identificada essa oportunidade de melhoria no processo. Essa é a motivação principal por trás da Extensionly. Desenvolver uma solução que contemple todos os processos envolvidos no ciclo de vida das atividades extensionistas. Para isso, o esforço conjunto de dois autores tem sido realizado, tanto na geração de artefatos de suporte à pesquisa, como no desenvolvimento da solução. Este trabalho tem como foco principal a parte do \textit{front-end} e experiência de usuário do sistema, enquanto que o outro concentra-se no \textit{back-end} da aplicação. Sobre os artefatos gerados, foram eles:
  \begin{inparaenum}[(a)]
    \item Um protocolo, formulado e executado para a realização de uma revisão sistemática na literatura cinza, de acordo com as diretrizes da Engenharia de Software, com o objetivo de encontrar ferramentas similares. Os resultados foram classificados e, a partir de sua análise, foi realizada uma extração de requisitos e necessidades iniciais da aplicação;
    \item Um \textit{survey}, cuja confecção foi realizada segundo definições e diretrizes encontradas na literatura. Esse estudo foi direcionado à comunidade acadêmica da \ac{UNIPAMPA} e teve como objetivo classificar, escala de importância, os requisitos previamente coletados com a revisão na literatura cinza. Os resultados foram analisados e, a partir deles, iniciou-se o desenvolvimento da solução proposta, uma solução \textit{web} para apoiar o processo de gestão dos programas e projetos de de extensão, cujos benefícios serão principalmente a redução de esforço necessário para a criação de uma atividade extensionista e a agilidade no engajamento dos extensionistas voluntários.
  \end{inparaenum}

  \vspace{\onelineskip}

  \noindent
  \textbf{Palavras-chave}: Ferramenta. Survey. Literatura Cinza. Frontend. Extensão. Universidade.
\end{resumo}

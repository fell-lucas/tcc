\begin{resumo}
 Em 2023, o processo de curricularização de novas Ações de Extensão será implantado obrigatoriamente pelas universidades. Apesar disso, essa gestão dos Programas e Projetos de Extensão continuaria sendo realizada manualmente pelo coordenador ou colaboradores da \acs{UNIPAMPA}. Essa é a motivação principal por trás da Extensionly. Desenvolver uma solução que contemple todos os processos envolvidos no ciclo de vida das atividades extensionistas. Para isso, um protocolo foi formulado e executado para a realização de um mapeamento sistemático na literatura cinza, de acordo com as diretrizes da Engenharia de Software, com o objetivo de encontrar ferramentas similares. Os resultados foram classificados e, a partir de sua análise, foi feita uma extração de requisitos e necessidades iniciais da aplicação. Posteriormente, foi feita a confecção de um survey segundo definições e diretrizes encontradas na literatura. Direcionado à comunidade acadêmica da \ac{UNIPAMPA}, teve como objetivo classificar, escala de importância, os requisitos previamente coletados com a revisão na literatura cinza. Os resultados foram analisados e, a partir deles, iniciou-se o desenvolvimento da solução proposta, uma aplicação web que suprirá as necessidades do público-alvo e reduzirá o esforço manual atualmente colocado nos processos de extensão.
 
%  Segundo a \citeonline[3.1-3.2]{NBR6028:2003}, o resumo deve ressaltar o
%  objetivo, o método, os resultados e as conclusões do documento. A ordem e a extensão
%  destes itens dependem do tipo de resumo (informativo ou indicativo) e do
%  tratamento que cada item recebe no documento original. O resumo deve ser
%  precedido da referência do documento, com exceção do resumo inserido no
%  próprio documento. (\ldots) As palavras-chave devem figurar logo abaixo do
%  resumo, antecedidas da expressão Palavras-chave:, separadas entre si por
%  ponto e finalizadas também por ponto.

 \vspace{\onelineskip}
    
 \noindent
 \textbf{Palavras-chave}: Ferramenta. Survey. Literatura Cinza. Frontend. Extensão. Universidade.
\end{resumo}

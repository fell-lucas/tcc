\begin{resumo}[Abstract]
In 2023, the process of curricularization of new outreach activities will be obligatorily implemented by universities in Brazil. Despite this, this management of outreach programs and projects would continue to be carried out manually by the coordinator or collaborators of \acs{UNIPAMPA}. This is the main motivation behind Extensionly. Develop a solution that includes all the processes involved in the life cycle of outreach activities. For this, a protocol was formulated and executed to carry out a systematic mapping in the gray literature, according to the guidelines of Software Engineering, in order to find similar tools. The results were classified and, from their analysis, an extraction of requirements and initial needs of the application was made. Subsequently, a survey was carried out according to definitions and guidelines found in the literature. Directed to the academic community of \ac{UNIPAMPA}, it aimed to classify, on a scale of importance, the requirements previously collected by reviewing the gray literature. The results were analyzed and, from them, the development of the proposed solution began, a web application that will meet the needs of the target audience and will reduce the manual effort currently placed in the outreach activities processes.

 \vspace{\onelineskip}
 
 \noindent 
 \textbf{Key-words}: Tool. Survey. Gray Literature. Frontend. Outreach Activities. University.
\end{resumo}

\begin{resumo}[Abstract]
  In 2023, the process of curricularization of new outreach actions will be implemented by the country's \ac{HEI}. Nevertheless, the Institutions do not have a completely automated process for the management of outreach programs and projects, which would continue to be carried out manually by the coordinator or outreach collaborators. Reality is no different in \acs{UNIPAMPA}, where this opportunity for improvement of the process was initially identified. This is the main motivation behind Extensionly. To develop a solution that contemplates all processes involved in the life cycle of outreach activities. For this, the joint effort of two authors has been made, both in the generation of research support artifacts and in developing the solution. This work has as its main focus in the front-end and system user experience, while the other focuses on the application back-end. About the artifacts generated, they were as follows:
  \begin{inparaenum}[(a)]
    \item A protocol, formulated and executed to perform a systematic review in grey literature, according to the software engineering guidelines, with the objective of finding similar tools. The results were classified and, from their analysis, an extraction of initial requirements and needs of the application was performed;
    \item A survey, whose confection was performed according to definitions and guidelines found in the literature. This study was directed to the academic community of \acs{UNIPAMPA} and aimed to classify, in the scale of importance, the requirements previously collected with the review in grey literature. The results were analyzed and, from them, the development of the proposed tool will start: A web solution to support the management of outreach programs and projects, whose benefits will be mainly the reduction of effort needed to create an outreach activity and agility in the engagement of volunteer outreach participants.
  \end{inparaenum}

  \vspace{\onelineskip}

  \noindent
  \textbf{Key-words}: Tool. Survey. Grey Literature. Front-end. Outreach Activities. University.
\end{resumo}

% Survey - lista preliminar da literatura -> validação com usuários reais, agregando novos requisitos
% protocolo, questionário, requisitos propostos, resultados, análise

%==============================================================================
\chapter{Survey}\label{survey}
%==============================================================================

\section{Survey Protocol}\label{sec:survey-protocol}

\subsection{Pilot Questionnaire}\label{sec:sv-p:pilot}

As \citeonline[p. 75]{kasunic2005designing} describes, a pilot test is a simulation of the real questionnaire carried out with a small number of members from the target audience. For this, the authors hand picked 7 (seven) people, out of which 4 (four) were students, 2 (two) were professors and 1 (one) was an \acl{TAE} (\ac{TAE}). The reason behind choosing this specific number of respondents is due to the following:
\begin{inparaenum}[(i)]
  \item All defined profiles for the respondents were chosen and
  \item the ratio of 4/2/1 is aligned with the expected numbers of submitted questionnaires per profile.
\end{inparaenum}

Unfortunately, the person chosen for the third profile, \ac{TAE}, wasn't able to answer. However, even though there are 3 (three) profiles, the questionnaire itself only has 2 (two) tracks of questions, one for students and the other for professors/\acp{TAE}. Because of that, the consequences of this happening weren't too impactful.

As for the pilot results, a lot of great feedback was received, along with some compliments on the organization of the questionnaire. There were issues with the person identification section, where the age was changed from a number to a range of numbers, such as between 21-29 years old.

\subsection{Distribute the Questionnaire}\label{sec:survey-distribute}

\section{Threats to validity}\label{sec:sv-validity}

% TAE didn't answer pilot test

\section{Results}\label{sec:sv-results}
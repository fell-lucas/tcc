% Survey - lista preliminar da literatura -> validação com usuários reais, agregando novos requisitos
% protocolo, questionário, requisitos propostos, resultados, análise

%=============================
\chapter{Survey}\label{survey}
%=============================

In this chapter, more detailed information is presented about the survey that was conducted. Similar to the previous chapter, which talks about the grey literature systematic review, the survey was also a joined effort work between two authors. The tasks on which each was responsible will be described later. The \Cref{sec:survey-protocol}, presents details about the protocol created, author of reference and division of tasks among the researchers. Afterwards, in \Cref{sec:survey-validity}, threats to the validity of the study are reported. Finally, \Cref{sec:survey-results}, presents all results achieved during execution.

\section{Survey Protocol}\label{sec:survey-protocol}

According to \citeonline{kasunic2005designing}, a survey is an approach to data collection and analysis in which participants answer questions or statements that were developed in advance. The protocol chosen for the elaboration of this study was also inspired by the guidelines proposed by the author and is illustrated in \Cref{fig:setepassos}.

\begin{figure}[!htb]
  \caption{Seven Steps of the Research Process}\label{fig:setepassos}
  \begin{center}
    \includegraphics[width=16cm]{img/5-kasunic-process.png}
  \end{center}
  \fonte{\cite{kasunic2005designing}.}
\end{figure}

As will be described in more detail later, the objective is to understand the needs of students and teachers in relation to projects and outreach activities. The choice of a survey as a data collection approach is due to the fact that according to \citeonline{kasunic2005designing}, the characteristics of such a research allows us to generalize about the beliefs and opinions of many people studying only a subset of them. Which is the perfect fit for this study.

Given that this research was performed by two students, the effort was divided equally, so that quality and performance were improved. \Cref{tbl:survey-tasks} describes the division of activities created by the authors and also already includes those defined by \citeonline{kasunic2005designing}.

\begin{table}[!htb]
  \centering
  \caption{Tasks Separation}
  \label{tbl:survey-tasks}
  \footnotesize
  \begin{tabular}{l|l}
    \bottomrule
    \rowcolor[rgb]{0.753,0.753,0.753} \multicolumn{1}{c|}{\textbf{Activity}}                             & \multicolumn{1}{c}{\textbf{Responsibility}} \\
    \hline
    \rowcolor[rgb]{0.898,0.898,0.898} Define and document research objectives                            & Lucas F.                                    \\
    Define and document research questions                                                               & Lucas F.                                    \\
    \rowcolor[rgb]{0.898,0.898,0.898} Define and document how research results will be used              & Lucas F.                                    \\
    Define the appropriate target audience for the research                                              & Igor C.                                     \\
    \rowcolor[rgb]{0.898,0.898,0.898} Determine the appropriate media to apply the research in           & Igor C.                                     \\
    Recruit members of the target audience to participate in pilot test                                  & Igor C.                                     \\
    \rowcolor[rgb]{0.898,0.898,0.898} Breakdown research questions into questionnaire topics             & Lucas F.                                    \\
    Organize and sequence questions                                                                      & Lucas F.                                    \\
    \rowcolor[rgb]{0.898,0.898,0.898} Review the questionnaire based on the pilot test                   & Igor C. and Lucas F.                        \\
    Perform the pilot test                                                                               & Igor C. and Lucas F.                        \\
    \rowcolor[rgb]{0.898,0.898,0.898} Evaluate comments                                                  & Igor C. and Lucas F.                        \\
    Perform final corrections before the distribution of the questionnaire                               & Lucas F.                                    \\
    \rowcolor[rgb]{0.753,0.753,0.753} \multicolumn{1}{c|}{\textbf{Questionnaire ready for distribution}} &                                             \\
    Distribute questionnaires                                                                            & Lucas F.                                    \\
    \rowcolor[rgb]{0.898,0.898,0.898} Monitor answers                                                    & Igor C. and Lucas F.                        \\
    Send reminders                                                                                       & Igor C.                                     \\
    \rowcolor[rgb]{0.753,0.753,0.753} \multicolumn{1}{c|}{\textbf{Questionnaire response deadline}}      &                                             \\
    Perform analysis                                                                                     & Igor C. and Lucas F.                        \\
    \rowcolor[rgb]{0.898,0.898,0.898} Write draft report                                                 & Igor C.                                     \\
    Revise draft                                                                                         & Igor C. and Lucas F.                        \\
    \rowcolor[rgb]{0.898,0.898,0.898} Perform the final corrections                                      & Igor C. and Lucas F.                        \\
    \toprule
  \end{tabular}
\end{table}

\subsection{Identify the Research Objectives}\label{sec:survey-objectives}

The point of having well defined research objectives, as \citeonline{kasunic2005designing} presents, is to increase the odds of executing a successful questionnaire. Through the results generated by the grey literature systematic review, mentioned earlier in \Cref{greyliterature}, it was possible to elaborate questions so that the participant informs, in his view, the importance of a certain requirement. This survey aims to order by priority and refine the elicited requirements, using the individual opinion the target audience.

In addition to being asked the participants' opinion, they were also allowed to provide long written feedbacks and suggestions or improvements to each of the presented requirements. Since one of the objectives of the survey was to refine existing requirements, enabling the users to describe their thoughts in more detail allowed the authors to identify underlying issues that would otherwise go unnoticed.

\subsection{Identify and Characterize the Target Audience}\label{sec:survey-targets}

In this stage, an analysis is made to identify possible target groups and to properly select the one that best fits with the research. The population is also defined, and is composed by the academic community as a whole. To represent the population sample, the outreach program and project coordinators, teachers and students, with a preference for participants who have experience with outreach activities, were chosen.

\subsection{Design the Sampling Plan}\label{sec:survey-sampling}

According to \citeonline{kasunic2005designing}, the purpose of this phase is to determine the following topics:
\begin{inparaenum}[(i)]
  \item How individuals will be selected to participate in the survey;
  \item The required size of the sample.
\end{inparaenum}

In order to select individuals to participate in the research, emails were sent to the Academic Secretariat of the \ac{UNIPAMPA} campuses, targeted to students and lists of outreach programs and projects coordinators. As expected, Uruguaiana and Bagé campuses, which executed the most outreach activities in 2021 \cite{relatorio-2017}, as seen in \Cref{fig:number-of-projects}, were the ones who provided the most respondents to the questionnaire.

\begin{figure}[!htb]
  \caption{Number of Projects Contemplated in the Internal Public Notices}\label{fig:number-of-projects}
  \begin{center}
    \includegraphics[width=16cm]{img/5-number-of-projects.pdf}
  \end{center}
  \fonte{Adapted from \citeonline{relatorio-2017}}
\end{figure}

Besides all quantitative answers, each respondent had the opportunity to discuss in more depth about the presented questions, allowing for a qualitative feedback, which increased significantly the effort required to make the analysis. In total, the questionnaire had 123 responses.

Sample separation is an essential point for the best efficiency of the survey. The approach chosen was the number 22, defined by \citeonline{Jefferson}, where the sample should be divided according to its characteristics and similarities. To implement it, the respondents of the questionnaire who declared themselves as \ac{ATE} or teachers were directed to one area of the questionnaire, and students to another, both areas with questions related to the responding profile.

\subsection{Design and Write the Questionnaire}\label{sec:survey-questionnaire}

According to \citeonline[p. 34]{kasunic2005designing}, questions that do not have well defined goals are more likely to have questions that only consume time from the respondent, he emphasizes this with the following question: ``How can you reach insightful conclusions if you do not know what you were looking for or planning to observe?''

In this survey the goal is well defined, focused on prioritization of requirements and suggestions by possible end users as described in \Cref{sec:survey-objectives}. Similarly, the characteristics of the sample are important to write the questions in a way that everyone understands and not just the researchers. \citeonline{surveyGuidelines} highlights that the results obtained with the survey are directly related to the quality of the questionnaire used.

For \citeonline{surveyGuidelines} there are two types of questionnaires:
\begin{inparaenum}[(i)]
  \item self-administrated and
  \item interviewer-administrated questionnaires.
\end{inparaenum}
This one fits the first type, because it is a web-based questionnaire. The researches don't have to monitor the respondents. This model allows for more respondents, but on the other hand tends to have a higher dropout rate, emphasizing the importance of good structuring.

Google Forms was the chosen tool to create the questionnaire, since it contributes with a simple and uncluttered interface, while also being a part of the Google Suite service, which is adopted by \ac{UNIPAMPA} to support various processes, such as institutional emails, for example. It is also widely used, being familiar to much of the respondents. The form structure can be seen in \Cref{appendix:questionnaire}. The next sections will briefly describe each part of the questionnaire.

\subsubsection{The Welcome Section}\label{survey:welcome}

Following instructions from \citeonline{kasunic2005designing}, the first page of the questionnaire contains important information for the participant, such as:

\begin{inparaenum}[(i)]
  \item Research objective;
  \item Estimated duration of the questionnaire;
  \item Researchers' contact email addresses;
  \item Researchers involved;
  \item Voluntary, anonymous and confidential character of the research;
  \item Institution and organization involved.
\end{inparaenum}

\subsubsection{Profile Questions}\label{survey:profile-questions}

The questions about the participant's personal information are important in the early stages of the questionnaire, as it motivates participants to continue answering the survey without asking complex questions early on \cite{LMRea}. In addition to a good classification of participants, it also allows the analysis of these to be done in a more controlled and organized way, as mentioned by \citeonline{legramante}.

The profile questions asked are listed below:
\begin{inparaenum}[(1)]
  \item Is enrolled in \ac{UNIPAMPA};
  \item Sex;
  \item Age group;
  \item Academic education;
  \item Already participated in an \ac{OA};
  \item Which roles the participant had in the \ac{OA};
  \item His role in the academic community;
  \item His campus and city;
  \item The course the participant is taking.
\end{inparaenum}

\subsubsection{Requirements Prioritization Questions}

In questions related to the research objective, some directions described by \citeonline{forza} were used, they are as follows:
\begin{inparaenum}[(1)]
  \item Define the way questions are asked to collect the information on a
  specific concept;\label{suggestion:1}
  \item For each question decide the scale on which the answers are placed;\label{suggestion:2}
  \item Identify the appropriate respondent(s) to each question;\label{suggestion:3}
  \item Put together the questions in questionnaires that facilitate and motivate the respondent(s) to respond.\label{suggestion:4}
\end{inparaenum}

\Cref{suggestion:1} suggests that the questions are written so that the entire responding sample can understand and formulate an answer. Since the questions of this questionnaire refer to software requirements, the user stories model has been used, which makes it very explicit who is the actor, what is desired with the requirement and the reason behind it. It was also determined that the questions would be classified as closed questions, where the possible answers are predetermined, as described by \citeonline{forza}. However, at the end of each page, an open-ended question was also described, allowing the respondent to write freely whichever thoughts he had.

\Cref{suggestion:2} is about the scale used in the questions. At first the Likert scale \cite{joshi2015likert} would be used, but after better analysis, it was decided to use the an adapted \ac{MoSCoW} scale, which is widely used in requirements prioritization \cite{waters2009prioritization}.

Afterwards, \Cref{suggestion:3} says that the questionnaire should direct the participants to the questions they have more property to answer, bringing more constructive and relevant answers. This division was made using the profile questions commented in \Cref{survey:profile-questions}, where the participant is automatically directed to the section corresponding with his profile.

Finally, \Cref{suggestion:4} suggests that all questions that have a common subject should be organized near each other to facilitate cross checks between each other. To implement this, the requirements were grouped by the actors' roles, and they are:
\begin{inparaenum}[(1)]
  \item \ac{OA} proponent;
  \item \ac{OA} instructor;
  \item \ac{OA} participant;
  \item Outreach programs and projects coordinator.
\end{inparaenum}

\subsubsection{Feature Suggestions}

For the last page of the questionnaire, a field was made available in which respondents may suggest to researchers any improvement, functionality, correction, anything they thought would be valuable for the goal product. With these answers it is possible to do a qualitative analysis and achieve new ideas for the development and completeness of the final tool.

\subsection{Pilot Questionnaire}\label{sec:survey-pilot}

As \citeonline{kasunic2005designing} describes, a pilot test is a simulation of the real questionnaire carried out with a small number of members from the target audience. For this, the authors arbitrarily invited 7 (seven) people, out of which 4 (four) were students, 2 (two) were professors and 1 (one) was an \ac{ATE}. The reason behind choosing this specific number of respondents is due to the following:
\begin{inparaenum}[(i)]
  \item All defined profiles for the respondents were chosen and
  \item the ratio of 4/2/1 is aligned with the expected numbers of submitted questionnaires per profile.
\end{inparaenum}

Unfortunately, the person chosen for the third profile, \ac{ATE}, wasn't able to answer. However, even though there are 3 (three) profiles, the questionnaire itself only has 2 (two) tracks of questions, one for students and the other for professors/\acp{ATE}. Because of that, the consequences of this happening weren't too impactful.

As for the pilot results, a lot of great feedback was received, along with some compliments on the organization of the questionnaire. There were issues with the person identification section, where the age was changed from a number to a range of numbers, such as between 19-29 years old.

\subsection{Distribute the Questionnaire}\label{sec:survey-distribute}

The questionnaire was distributed to all people who make up the sample of this research. For this, first was collected all emails of coordinators with active outreach projects or programs, from several campuses of \ac{UNIPAMPA}. They were the first to respond the questionnaire.

After two (2) days, emails were sent to all campus academic secretariats, requesting that it be passed on to all the students from all courses. In total, the survey was open to answers for eighteen (18) days.

\subsection{Analyze the Results and Write a Report}\label{sec:survey-analyse}

The quantitative results related to the prioritization of requirements were collected and organized in graphs to better understand and visualize the data, while the qualitatives were subjectively analyzed and incorporated into the refined requirements list. Thus it will be possible to have an orderly list of requirements that were considered most important to end users, as well as well described user stories.

\section{Threats to validity}\label{sec:survey-validity}



% ATE didn't answer pilot test

\section{Results}\label{sec:survey-results}
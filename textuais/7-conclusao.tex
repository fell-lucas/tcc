% Considerações Preliminares - possível publicação em eventos da área, com os resultados encontrados (ERES agosto), mais eventos (SBIE), artigos relacionados à extensão
%==============================================================================
\chapter{Preliminary Considerations}\label{conclusao}
%==============================================================================

% Em Trabalhos de Conclusão de Curso, use ``\emph{Considerações Finais}'' e não ``\emph{Conclusão}''.

% Bom trabalho!

The goal of the current work was to thoroughly examine the administration of \aclp{OA} in order to create a tool that supports the administrative procedures, makes it easier to communicate with the outside world, and improves student participation and dissemination. For this, two artifacts were produced: a review of the gray literature to locate comparable solutions already in use, eliciting their critical features, and a survey of potential end users of the tool to order the requirements by importance and gather additional ideas on the topic.

Five particular goals were established in order to accomplish the overall goal of the work, which is to create the tool's front-end to support the management of \ac{UNIPAMPA}'s outreach programs and projects. They were previously described in \Cref{sec:objectives}.

The first goal was to conduct a review of the gray literature to look for features in tools that were similar to what is being proposed. Based on the results, it is clear that this goal was accomplished, as it was possible to identify a number of tools and, in the end, to produce a list of tools that was significant in size, complete with their functionalities and specific information, which was especially helpful in planning.

The second goal relates to the development of a survey to ascertain the viewpoints of potential end users. This goal was accomplished since it was feasible to examine the problems and worries that users had, which made it possible to draw out a number of enhancements and features for the suggested tool.

The third particular goal is the creation of a development roadmap and tangible tasks. This goal was only partially attained because the transition from \ac{FR} to development tasks won't take place until the second phase of this work's development, or \ac{TP} II.

The fourth goal is to research, evaluate, and select a development stack, including a programming language, an architecture, and a framework, for the front-end of the suggested tool. This goal was accomplished; prior decisions were presented earlier in \Cref{extensionly}.

The creation of a \ac{MVP} for the tool's front-end is the fifth and final specific goal, and it has not yet been accomplished because the tool's development is still on the horizon.

The hypothesis of this work, which can't yet be confirmed or disproved because the application hasn't yet been developed and tested with end users, is that "With a tool to support the management of outreach programs and projects, it's possible to have a reduction on the effort needed to create an outreach activity and an increase in the engagement of volunteer outreach participants." However, given all of the good feedback that the survey respondents supplied, it is extremely probable that this theory will be proven with the accurate and thorough development of the application.

With a tool of this kind, it is feasible to lessen the manual and repetitive work required in the registration of new proposals for outreach efforts, facilitating operations like generating certificates more effectively. This is the research question of this work, described in \Cref{tbl:intro-objectives}. Additionally, students will know where to go if they need assistance with the subject thanks to a platform that centralizes information about university outreach, which will increase the dissemination of new initiatives. With the use of the technology, the connection between the teacher and the participant will also be reinforced, enabling richer interactions and experiences for both parties.

The survey's results gave the researchers insight into what potential end users could think about the presence of a tool like this to help them with related tasks. It allowed for the reevaluation of various implementation-related difficulties while taking into consideration suggestions made. It is safe to state that without a study of these users' opinions, the tool would be in serious danger of not meeting of the previously established expectations.

A lot of effort will be devoted to the application's development for the second iteration of this \ac{TP} so that real-world use case scenarios involving professors and students can be tested. The goal is to demonstrate the tool's value to the university as a whole by integrating its capabilities into actual outreach initiatives and projects from \ac{UNIPAMPA}.
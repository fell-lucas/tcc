% Background - extensão universitária no Brasil, curricularização da extensão, soluções/ferramentas de apoio à extensão, leis federais, resoluções unipampa, implantação da extensão, tipos de extensão, perfis de pessoas envolvidas na extensão, programas e projetos de extensão na unipampa
% Unipampa Cidadã
%==============================================================================
\chapter{Background}\label{background}
%==============================================================================

In this chapter, information that complement the objective of the study are discussed, helping to understand the policies and resolutions involved. In \Cref{sec:bac-1} the national outreach activity policy will be presented, which is valid for all universities in Brazil. It applies for each \ac{OA} regarding its relation to the academic and external community. Soon after in \Cref{sec:bac-2-0} and \Cref{sec:bac-2} the vision of how both the \ac{ICES} as a whole and \acl{UNIPAMPA}, respectively, adapted to receive these new rules is described. Afterwards, in \Cref{sec:bac-3} the differences between outreach programs and projects will be presented, followed by a more detailed explanation about the ``\ac{UNIPAMPA} Cidadã'' project in \Cref{sec:bac-4}. The \Cref{sec:bac-5} showcases current available tools and solutions in the market which are related to the study goal product. The \Cref{sec:bac-6} reveals some tools related to the subject of the work, their commonalities and a high-level description. Finally in \Cref{sec:bac-7} a general summary of the chapter is presented.

\section{Outreach activities in Brazil}\label{sec:bac-1}

It is clear that participating in outreach activities has many benefits for the students who decide to take part in it \cite{sellou2011many}. Besides promoting individual growth, the activities can also serve as a bridge connecting students and professors even more. In order to preserve them and encourage younger students to participate in them, the \acl{FORPROEX} (\ac{FORPROEX}), updated the old version of the National Outreach Policy document, published in 1999, with current situations and challenges encountered in recent years. In the new version of the document, \cite{politicaNacional}, some of its objectives are the following:

\begin{itemize}
  \item Achieve the recognition of university outreach activities as an essential tool for the public university.
  \item Ensure that the outreach activity is the solution to any type of social problem faced by the country.
  \item Defend the funding of outreach programs and projects so that they can continue to function.
  \item Promote environmental and sustainable awareness in outreach projects in Brazil.
  \item Promote solidarity both nationally and internationally, covering the area of impact of outreach actions.
\end{itemize}

As a reference for directing and assisting \aclp{ICES} (\ac{ICES}) to create their outreach policies, \cite{referenciaisPolitica} also highlights the importance of integrating outreach activities with research and teaching, along with discussions of a social nature and the effects of the results on society. The document proposes nine outreach activity types, which are as follows:

\begin{inparaenum}[(1)]
  \item Programs, Projects and Activities for the socialization of knowledge;
  \item Outreach Courses;
  \item Participation in Councils, Academic Events open to the external community: Congresses, Symposia, Seminars, Colloquiums, Course Weeks and related activities;
  \item Promotions of Art, Culture, Sport and Leisure with the involvement of the external community;
  \item Provision of Services, Consultancy and Advisory Services, Technological Extension, Mandatory Internships;
  \item School Clinics;
  \item Curricular Professional Practices;
  \item Disciplines that include practices with external communities;
  \item Research Projects, Course Completion Works,
  Monographs, Dissertations and Theses with methodologies and practices of social intervention with external communities.
\end{inparaenum}

\subsection{\acl{OA} curricularization in Higher Education}\label{sec:bac-2-0}

In order to implement what was mentioned above in the \ac{ICES}, the Brazilian Ministry of Education created the Resolution No. 7, of December 18, 2018, which establishes guidelines, principles, foundations and procedures for \acp{OA} in higher education. As such, it was regulated that \acp{OA} will be made available in the form of curricular components for the offered courses \cite{ministerioSuperiorExtensao}.

The document also determines that the outreach activities must comprise at least 10\% (ten percent) of the total student curricular workload of undergraduate courses, and they must also be part of the curriculum of the courses \cite[p. 2, art. 4]{ministerioSuperiorExtensao}. Another important discussed topic is about the self-assessment of \acp{OA}. The main reason for this is the constant improvement of the activity with teaching, research, student training, teacher qualification, the relationship with society, the participation of partners and other institutional academic dimensions.. This evaluation must include the following:

\begin{inparaenum}[(a)]
  \item How many curricular credits the activity can give;
  \item How it contributes to the Institutional Development Plan and the Pedagogical Projects for the Courses;
  \item The demonstration of the results achieved in relation to the participating public.
\end{inparaenum}

Each \ac{OA} must also contain the planning of its internal activities, the strategies for self-assessment, proposal, development and conclusion. These must be duly recorded and analyzed in order to organize the activity work plans.

As a final note, the Resolution says that the higher education instutitions will have at most 3 (three) years, counting by the date the document was published, to implement what is being proposed.

\subsection{\acl{OA} curricularization in \acl{UNIPAMPA}}\label{sec:bac-2}

In relation to \ac{UNIPAMPA}, as with other \ac{ICES}, it must create a resolution aimed at standardizing outreach activities in general, presenting what they are, their target audience and their objectives. And thus was born the CONSUNI/UNIPAMPA Resolution No. 332 of 2021, which determines the types of outreach activities, already mentioned earlier in the study, their managing bodies, executing team, possible related processes, and rules such as the minimum duration of 8 (eight) hours \cite{Resolucao-332:2021}.

As \ac{UNIPAMPA} highlights in the Resolution No. 317 of 2021, the main objectives in the insertion of outreach activities in undergraduate courses are the following \cite{res317}:
\begin{inparaenum}[(i)]
  \item Help students develop their critical, civic, interdisciplinary and responsible education;
  \item Improve teaching in undergraduate courses as a whole and strengthen the inseparability between teaching, research and outreach;
  \item Strengthen \ac{UNIPAMPA}'s social commitment;
  \item Stimulate constructive discussions in all sectors of \ac{UNIPAMPA};
  \item Promote actions that strengthen \ac{UNIPAMPA}'s ethical principles and social commitment in all areas;
  \item Encourage the academic community to be more present in human, academic, social, cultural and economic development.
\end{inparaenum}

\subsection{Outreach Programs and Projects}\label{sec:bac-3}

According to \citeonline{referenciaisPolitica}, Outreach Program and Projects

\subsection{UNIPAMPA Cidadã}\label{sec:bac-4}

\section{User profiles}\label{sec:bac-5}

\section{Similar tools}\label{sec:bac-6}

\section{Chapter Summary}\label{sec:bac-7}
%% monografia.tex, fabiokepler, jeancheiran
%% Copyright 2012-2018 by UNIPAMPA LaTeX group at https://bitbucket.org/unipampaalegrete/monografias-cc-es-repo/
%%
%% This work may be distributed and/or modified under the conditions of the LaTeX Project Public
%% License, either version 1.3 of this license or (at your option) any later version.
%% The latest version of this license is in
%%   http://www.latex-project.org/lppl.txt
%% and version 1.3 or later is part of all distributions of LaTeX version 2005/12/01 or later.
%%
%% Based on the example file abtex2-modelo-trabalho-academico.tex of the abntex2 package
%% (http://abntex2.googlecode.com/) and on the ppgccufmg 1.45beta2 class
%% (http://vilarneto.com/ppgccufmg,
%% http://www.dcc.ufmg.br/pos/alunos/modelodisstese.php
%% and http://www.dcc.ufmg.br/~mirella).
%%
%% Adapted for the Computer Science program at UNIPAMPA (http://www.unipampa.edu.br)
%% by Fabio Kepler (fabio@kepler.pro.br) and Jean Cheiran (jeancheiran@unipampa.edu.br).
%%
%% Version 2.5 - 2018/08
%% Version 2.4 - 2017/05
%% Version 2.3 - 2013/03

% +++++++++++++++++++++++++++++++++++++++++++++++++++++++++++++++++++++++++++++++++++++++++++++++++
% Este modelo utiliza o pacote abnTeX2. Veja como instalá-lo em seu ambiente em
% http://abntex2.googlecode.com/.
% -------------------------------------------------------------------------------------------------
% abnTeX2: Modelo de Trabalho Acadêmico (tese de doutorado, dissertação de
% mestrado e trabalhos monográficos em geral) em conformidade com
% ABNT NBR 14724:2011: Informação e documentação - Trabalhos acadêmicos -
% Apresentação
% -------------------------------------------------------------------------------------------------
% Normas institucionais utilizadas:
% http://porteiras.r.unipampa.edu.br/portais/sisbi/programa-de-capacitacao/
% +++++++++++++++++++++++++++++++++++++++++++++++++++++++++++++++++++++++++++++++++++++++++++++++++

\documentclass[12pt,openright,twoside,a4paper,chapter=TITLE]{abntex2}    % frente e verso
%\documentclass[12pt,oneside,a4paper]{abntex2}            % apenas frente

% +++++++++++++++++++++++++++++++++++++++++++++++++++++++++++++++++++++++++++++++++++++++++++++++++
% PACOTES
% -------------------------------------------------------------------------------------------------
% Pacotes fundamentais
\usepackage{cmap}           % Mapeamento de caracteres especiais no PDF
\usepackage{lmodern}        % Usa fonte Latin Modern
\usepackage[T1]{fontenc}    % Seleção de codificação de fonte
\usepackage[utf8]{inputenc} % Codificação do arquivo (conversão automática dos acentos)
\usepackage{makeidx}        % Criação de índice
\usepackage{hyperref}       % Formatação do índice
\usepackage{lastpage}       % Usado pela Ficha catalográfica
\usepackage{indentfirst}    % Indenta o primeiro parágrafo de cada seção
\usepackage[usenames,dvipsnames]{xcolor}  % Controle das cores (com nomes)
\usepackage{graphicx}       % Inclusão de gráficos
\usepackage{booktabs}       % Formatação de tabelas
% -------------------------------------------------------------------------------------------------
% Para citações
\usepackage[brazilian,hyperpageref]{backref} % Páginas com as citações na bibliografia
\usepackage[alf,abnt-emphasize=bf]{abntex2cite} % Citações padrão ABNT (alfanumérico)
% -------------------------------------------------------------------------------------------------
% Pacotes opcionais
\usepackage{nomencl}        % Para criar uma lista de símbolos
\usepackage{acro}           % Para usar acrônimos e abreviaturas
\usepackage{tikz}           % Para fazer figuras, diagramas e gráficos integrados e elegantes
\usepackage{pgfplots}       % Usa o pacote tikz para fazer gráficos muito melhores que os do Excel
\usepackage{pgfplotstable}  % Para gerar tabelas automaticamente a partir de arquivos com dados
\usepackage{filecontents}   % Para colocar o conteúdo de um arquivo dentro de um arquivo tex
%\usepackage{multirow}       % Permite fazer tabelas com múltiplas linhas
%\let\newfloat=\undefined    % Workaround para usar o pacote algorithm
%\usepackage{algorithm}      % Para escrever algoritmos
%\usepackage{clrscode}       % Para escrever algoritmos
%\usepackage{clrscode3e}     % Para escrever algoritmos; mais simples que os pacotes acima
\usepackage{pdfpages}        % Para incluir a folha de aprovação assinada em PDF
\usepackage[olditem]{paralist}       % Listas in-line
% -------------------------------------------------------------------------------------------------
\addto\captionsenglish{% ingles
  %% adjusts names from abnTeX2
  \renewcommand{\folhaderostoname}{Title page}
  \renewcommand{\epigraphname}{Epigraph}
  \renewcommand{\dedicatorianame}{Dedication}
  \renewcommand{\errataname}{Errata sheet}
  \renewcommand{\agradecimentosname}{Acknowledgements}
  \renewcommand{\anexoname}{ANNEX}
  \renewcommand{\anexosname}{Annex}
  \renewcommand{\apendicename}{APPENDIX}
  \renewcommand{\apendicesname}{Appendix}
  \renewcommand{\orientadorname}{Supervisor:}
  \renewcommand{\coorientadorname}{Co-supervisor:}
  \renewcommand{\folhadeaprovacaoname}{Approval}
  \renewcommand{\resumoname}{Resumo} 
  \renewcommand{\listadesiglasname}{List of abbreviations and acronyms}
  \renewcommand{\listadesimbolosname}{List of symbols}
  \renewcommand{\fontename}{Source}
  \renewcommand{\notaname}{Note}
   %% adjusts names used by \autoref
  \renewcommand{\pageautorefname}{page}
  \renewcommand{\sectionautorefname}{section}
  \renewcommand{\subsectionautorefname}{subsection}
  \renewcommand{\subsubsectionautorefname}{subsubsection}
  \renewcommand{\paragraphautorefname}{subsubsubsection}
  \renewcommand{\englishbibname}{Bibliography}
  \renewcommand{\englishindexname}{Index}
  \renewcommand{\englishlistfigurename}{List of figures}
  \renewcommand{\englishfigurename}{Figure}
  \renewcommand{\englishlisttablename}{List of tables}
  \renewcommand{\englishtablename}{Table}
  \renewcommand{\englishcontentsname}{List of contents}
  \renewcommand{\englishchaptername}{Chapter}
  \renewcommand{\imprimirtipotrabalho}{Term Paper }
}
% Configurações do pacote backref
% Usado sem a opção hyperpageref de backref
\renewcommand{\backrefpagesname}{Cited in page(s):~}
% Texto padrão antes do número das páginas
\renewcommand{\backref}{}
% Define os textos da citação
\renewcommand*{\backrefalt}[4]{
    \ifcase #1 %
        No text citation.%
    \or
        Cited in page #2.%
    \else
        Cited #1 times on pages #2.%
    \fi}%
% -------------------------------------------------------------------------------------------------
% Configurações de aparência do PDF final
%\definecolor{blue}{RGB}{41,5,195}
% \definecolor{webgreen}{rgb}{0,.5,0}
% Metainformações do PDF e cores dos links
\hypersetup{
  portuguese,
  %backref=true,
  %pagebackref=true,
  %bookmarks=true,             % show bookmarks bar?
  %bookmarksnumbered=true,
  bookmarksdepth=4,
  pdftitle={\@title},
  pdfauthor={\@author},
  pdfsubject={\imprimirpreambulo},
  pdfkeywords={UNIPAMPA}{Computação}{UNIPAMPA}{abntex}{TCC},
  %pdfproducer={LaTeX with abnTeX2},     % producer of the document
  pdfcreator={\@author},
  colorlinks=true,           % false: boxed links; true: colored links
  linkcolor=black,            % color of internal links
  citecolor=black,            % color of links to bibliography
  filecolor=black,         % color of file links
  urlcolor=black
}
%   linktocpage,
%   colorlinks,
%   citecolor=webgreen,
%   urlcolor=Maroon,
%   linkcolor=RoyalBlue,
%   filecolor=black,
% -------------------------------------------------------------------------------------------------
% Espaçamentos entre linhas e parágrafos
% O tamanho do parágrafo é dado por
\setlength{\parindent}{1.3cm}
% Controle do espaçamento entre um parágrafo e outro
\setlength{\parskip}{0.2cm} % tente também \onelineskip
% Controles do espaçamento entre linhas
%\OnehalfSpacing       % espaçamento um e meio (padrão);
%\DoubleSpacing        % espaçamento duplo
%\SingleSpacing        % espaçamento simples
% -------------------------------------------------------------------------------------------------
% Para o pacote de acrônimos
\acsetup{make-links} %first-style=short}
% -------------------------------------------------------------------------------------------------
% Para o pacote tikz, pgfplots e pgfplotstable
\usetikzlibrary{arrows,chains,matrix,positioning,decorations.pathreplacing,calc}
% -------------------------------------------------------------------------------------------------
% Para poder usar subfiguras e subtabelas
\newsubfloat{figure}
\newsubfloat{table}
\providecommand*{\subfigureautorefname}{\figureautorefname}
% +++++++++++++++++++++++++++++++++++++++++++++++++++++++++++++++++++++++++++++++++++++++++++++++++


% +++++++++++++++++++++++++++++++++++++++++++++++++++++++++++++++++++++++++++++++++++++++++++++++++
% Informações de dados para CAPA e FOLHA DE ROSTO
% -------------------------------------------------------------------------------------------------
\titulo{Extensionly - A tool for supporting the management of extracurricular projects and programs: Frontend}
\autor{Lucas Alexandre Fell}
\local{Alegrete}
\data{2022}
\orientador{Prof. PhD. Maicon Bernardino da Silveira}
\instituicao{Federal University of Pampa}
%\tipotrabalho{Projeto de Trabalho de Conclusão de Curso~} % Para TCC I
\tipotrabalho{Trabalho de Conclusão de Curso~} % Para TCC II
% O preambulo deve conter o tipo do trabalho, o objetivo, o nome da instituição e a área de concentração
\preambulo{\imprimirtipotrabalho presented in Software Engineering Graduation Course in the Federal University of Pampa as a partial requirement for
obtaining the title of Software Engineering
Bachelor}


\makeindex % Compila o indice
\makenomenclature % Compila a lista de abreviaturas e siglas
% Abreviaturas (definido pelo parâmetro 'class')
\DeclareAcronym{fig}{
  short = Fig.,
  long  = Figura,
  tag = abreviaturas
}
% -------------------------------------------------------------------------------------------------
% Acrônimos/Siglas (definido pelo parâmetro 'class')
\DeclareAcronym{UNIPAMPA}{
  short = UNIPAMPA,
  long  = Federal University of Pampa,
  tag = acronimos
}
\DeclareAcronym{OCA}{
  short = OCA,
  long  = Outreach Curriculum Activity,
  long-plural-form = Outreach Curriculum Activities,
  tag = acronimos
}
\DeclareAcronym{PROEXT}{
  short = PROEXT,
  long  = Dean of Extension and Culture,
  tag = acronimos
}
\DeclareAcronym{IT}{
  short = IT,
  long  = Information Technology,
  tag = acronimos
}
\DeclareAcronym{MVP}{
  short = MVP,
  long  = Minimum Viable Product,
  tag = acronimos
}

\begin{document}
% Pequenos consertos e ajustes para que fique de acordo com o Manual de Normatização 2011.

\setlength{\ABNTEXsignwidth}{12cm}

%% adjusts names from abnTeX2
% \renewcommand{\folhaderostoname}{Title page}
% \renewcommand{\epigraphname}{Epigraph}
% \renewcommand{\dedicatorianame}{Dedication}
% \renewcommand{\errataname}{Errata sheet}
% \renewcommand{\agradecimentosname}{Acknowledgements}
% \renewcommand{\anexoname}{ANNEX}
% \renewcommand{\anexosname}{Annex}
% \renewcommand{\apendicename}{APPENDIX}
% \renewcommand{\apendicesname}{Appendix}
% \renewcommand{\orientadorname}{Supervisor:}
% \renewcommand{\coorientadorname}{Co-supervisor:}
% \renewcommand{\folhadeaprovacaoname}{Approval}
% % \renewcommand{\resumoname}{Resumo
% % \renewcommand{\englishcontentsname}{Summary}
% % \renewcommand{\englishlistfigurename}{List of figures}
% % \renewcommand{\englishlisttablename}{List of tables}
% % \renewcommand{\englishlistadesiglasname}{List of abbreviations and acronyms}
% \renewcommand{\listadesimbolosname}{List of symbols}
% \renewcommand{\fontename}{Source}
% \renewcommand{\notaname}{Note}
% %% adjusts names used by \autoref
% \renewcommand{\pageautorefname}{page}
% \renewcommand{\sectionautorefname}{section}
% \renewcommand{\subsectionautorefname}{subsection}
% \renewcommand{\subsubsectionautorefname}{subsubsection}
% \renewcommand{\paragraphautorefname}{subsubsubsection}
% \renewcommand{\imprimirtipotrabalho}{Term Paper }

% ---
% Impressão da Capa
\renewcommand{\imprimircapa}{%
  \begin{capa}%
    \center
    {\ABNTEXchapterfont\large\MakeUppercase\imprimirinstituicao}

    \vspace*{\fill}
    {\ABNTEXchapterfont\large\imprimirautor}

    \vspace*{\fill}
    {\ABNTEXchapterfont\bfseries\LARGE\imprimirtitulo}

    \vspace*{\fill}
    ~
    \vspace*{\fill}

    {\large\imprimirlocal}
    \par
    {\large\imprimirdata}

    \vspace*{1cm}
  \end{capa}
}
% ---


% ---
% Impressão da Folha de Rosto
\makeatletter
\renewcommand{\folhaderostocontent}{
  \begin{center}

    {\ABNTEXchapterfont\large\imprimirautor}

    \vspace*{\fill}%\vspace*{\fill}
    {\ABNTEXchapterfont\bfseries\Large\imprimirtitulo}
    \vspace*{\fill}

    \abntex@ifnotempty{\imprimirpreambulo}{%
      \hspace{.45\textwidth}
      \begin{minipage}{.5\textwidth}
        {\SingleSpacing
        \imprimirpreambulo}

        \vspace*{1em}
        \imprimirorientadorRotulo~\imprimirorientador\par

        \abntex@ifnotempty{\imprimircoorientador}{%
          \vspace*{1em}
          \imprimircoorientadorRotulo~\imprimircoorientador%
        }%

      \end{minipage}%
      \vspace*{\fill}
    }%

    {\large\imprimirlocal}
    \par
    {\large\imprimirdata}
    \vspace*{1cm}

  \end{center}
}
\makeatother
% ---

% ---
\renewcommand{\ABNTEXchapterfont}{\rmfamily\bfseries}
\setsecheadstyle{\rmfamily\bfseries}

\renewcommand{\ABNTEXchapterfontsize}{\normalsize}
\renewcommand{\ABNTEXsectionfontsize}{\normalsize}
\renewcommand{\ABNTEXsubsectionfontsize}{\normalsize}
\renewcommand{\ABNTEXsubsubsectionfontsize}{\normalsize}
\renewcommand{\ABNTEXsubsubsubsectionfontsize}{\normalsize}

% Espaçamento entre título e texto
\setlength\afterchapskip{\lineskip}

% Espaçamento entre parágrafos
\setlength{\parskip}{0.cm}

% ---

 % Inclui alguns ajustes finos para que fique de acordo com o Manual de Normatização
\selectlanguage{english}

% ELEMENTOS PRÉ-TEXTUAIS
\imprimircapa % Capa [OBRIGATÓRIO]
\imprimirfolhaderosto % Folha de rosto [OBRIGATÓRIO]

% -----------------------------------------------
% Folha de aprovação [OBRIGATÓRIO]
% -----------------------------------------------
% Este é um exemplo de Folha de aprovação, elemento obrigatório da NBR 14724/2011 (seção 4.2.1.3).
% Você pode utilizar este modelo até a aprovação do trabalho.
% Após isso, altere o conteúdo deste arquivo para inserir uma imagem da página assinada pela banca usando
% o modelo que está no final deste arquivo.

% -----------------------------------------------
% Folha de aprovação antes da defesa do TCC
% -----------------------------------------------
%\begin{comment}
\begin{folhadeaprovacao}
  \begin{center}
    {\ABNTEXchapterfont\large\imprimirautor}

    \vspace*{\fill}%\vspace*{\fill}
    {\ABNTEXchapterfont\bfseries\Large\imprimirtitulo}
    \vspace*{\fill}

    \hspace{.45\textwidth}
    \begin{minipage}{.5\textwidth}
        \imprimirpreambulo
    \end{minipage}%
    \vspace*{\fill}
  \end{center}

  \begin{center}
    \imprimirtipotrabalho presented and approved on ..... .............. of ......

    Committee members:
  \end{center}

  \assinatura{\textbf{\imprimirorientador} \\ Supervisor \\ UNIPAMPA}
  \makeatletter
  \abntex@ifnotempty{\imprimircoorientador}{%
    \assinatura{\textbf{\imprimircoorientador} \\ Coorientador \\ <sigla da instituição>}%
  }
  \makeatother
  \assinatura{\textbf{Prof. <titulação> Nome Professor} \\ <sigla da instituição>}
  \assinatura{\textbf{Prof. <titulação> Nome Professor} \\ <sigla da instituição>}

\end{folhadeaprovacao}
%\end{comment}
% -----------------------------------------------
% Folha de aprovação após a defesa do TCC com a imagem da folha de aprovação assinada pela banca.
% -----------------------------------------------
\begin{comment}
\begin{folhadeaprovacao}

% Escolher entre uma das seguintes opções para inclusão da folha de aprovação
% Versão assinada em arquivo PDF (incluir no arquivo principal o comando \usepackage{pdfpages})
%\includepdf{pretextuais/aprovacao.pdf}

% Ou, versão assinada em arquivo de imagem (jpg, png, etc)
% Mas prefira em PDF. Em imagem é preciso acertar os recuos das margens:
%\vspace*{-4cm}
%\hspace*{-3.5cm}
%\includegraphics[width=\paperwidth]{pretextuais/aprovacao}

\end{folhadeaprovacao}
\end{comment}
 % Folha de aprovação [OBRIGATÓRIO]
\begin{dedicatoria}
   \vspace*{\fill}
   \begin{flushright}
     This work is dedicated to all software engineering empiricists who,\\
     at some point, felt like giving up\\
     and throwing everything up in the air,\\
     but still made it to the end.
   \end{flushright}
   \vspace*{\fill}
\end{dedicatoria}
 % Dedicatória [OPCIONAL]
\begin{agradecimentos}

I would like to thank my family, Isabel, Marco and Maitê for their unbounded love and support. I wouldn't  be here without their help throughout the years. 

I am also grateful for the knowledge and education I received from each of my professors during my time at the University. This work wouldn't be possible without them.

And my advisor, Maicon Bernardino, for his guidance and patience.

\end{agradecimentos} % Agradecimentos [OPCIONAL]
\begin{epigrafe}
  \vspace*{\fill}
	\begin{flushright}
		\textit{``The most beautiful experience we can have is the mysterious.\\
		It is the fundamental emotion that stands at the cradle of true art and true science.''
		(Albert Einstein)}
	\end{flushright}
\end{epigrafe}
 % Epígrafe [OPCIONAL]
\begin{resumo}
  Em 2023, o processo de curricularização de novas Ações de Extensão será implantado obrigatoriamente pelas Instituições de Ensino Superior do país. Apesar disso, em sua maioria, as Instituições não possuem um processo completamente automatizado para a gestão dos Programas e Projetos de Extensão, que continuaria sendo realizada manualmente pelo coordenador ou colaboradores de extensão. A realidade não é diferente na \acs{UNIPAMPA}, onde foi inicialmente identificada essa oportunidade de melhoria no processo. Essa é a motivação principal por trás da Extensionly. Desenvolver uma solução que contemple todos os processos envolvidos no ciclo de vida das atividades extensionistas. Para isso, o esforço conjunto de dois autores tem sido realizado, tanto na geração de artefatos de suporte à pesquisa, como no desenvolvimento da solução. Este trabalho tem como foco principal a parte do \textit{front-end} e experiência de usuário do sistema, enquanto que o outro concentra-se no \textit{back-end} da aplicação. Sobre os artefatos gerados, foram eles:
  \begin{inparaenum}[(a)]
    \item Um protocolo, formulado e executado para a realização de uma revisão sistemática na literatura cinza, de acordo com as diretrizes da Engenharia de Software, com o objetivo de encontrar ferramentas similares. Os resultados foram classificados e, a partir de sua análise, foi realizada uma extração de requisitos e necessidades iniciais da aplicação;
    \item Um \textit{survey}, cuja confecção foi realizada segundo definições e diretrizes encontradas na literatura. Esse estudo foi direcionado à comunidade acadêmica da \ac{UNIPAMPA} e teve como objetivo classificar, escala de importância, os requisitos previamente coletados com a revisão na literatura cinza. Os resultados foram analisados e, a partir deles, iniciou-se o desenvolvimento da solução proposta, uma solução \textit{web} para apoiar o processo de gestão dos programas e projetos de de extensão, cujos benefícios serão principalmente a redução de esforço necessário para a criação de uma atividade extensionista e a agilidade no engajamento dos extensionistas voluntários.
  \end{inparaenum}

  \vspace{\onelineskip}

  \noindent
  \textbf{Palavras-chave}: Ferramenta. Survey. Literatura Cinza. Frontend. Extensão. Universidade.
\end{resumo}
 % Resumo [OBRIGATÓRIO]
\begin{resumo}[Abstract]
  In 2023, the process of curricularization of new outreach actions will be implemented by the country's \ac{HEI}. Nevertheless, the Institutions do not have a completely automated process for the management of outreach programs and projects, which would continue to be carried out manually by the coordinator or outreach collaborators. Reality is no different in \acs{UNIPAMPA}, where this opportunity for improvement of the process was initially identified. This is the main motivation behind Extensionly. To develop a solution that contemplates all processes involved in the life cycle of outreach activities. For this, the joint effort of two authors has been made, both in the generation of research support artifacts and in developing the solution. This work has as its main focus in the front end and system user experience, while the other focuses on the application back end. About the artifacts generated, they were as follows:
  \begin{inparaenum}[(a)]
    \item A protocol, formulated and executed to perform a systematic review in grey literature, according to the software engineering guidelines, with the objective of finding similar tools. The results were classified and, from their analysis, an extraction of initial requirements and needs of the application was performed;
    \item A survey, whose confection was performed according to definitions and guidelines found in the literature. This study was directed to the academic community of \acs{UNIPAMPA} and aimed to classify, in the scale of importance, the requirements previously collected with the review in grey literature. The results were analyzed and, from them, the development of the proposed tool will start: A web solution to support the management of outreach programs and projects, whose benefits will be mainly the reduction of effort needed to create an outreach activity and agility in the engagement of volunteer outreach participants.
  \end{inparaenum}

  \vspace{\onelineskip}

  \noindent
  \textbf{Key-words}: Tool. Survey. Grey Literature. Front end. Outreach Activities. University.
\end{resumo}
 % Abstract (resumo em inglês) [OBRIGATÓRIO]
% Figuras/Ilustrações [OPCIONAL]
\pdfbookmark[0]{\listfigurename}{lof}
\listoffigures*
\cleardoublepage

% Tabelas [OPCIONAL]
\pdfbookmark[0]{\listtablename}{lot}
\listoftables*
\cleardoublepage

% Siglas [OPCIONAL] (veja o pacote acro e os exemplo acima)
\pdfbookmark[0]{\listadesiglasname}{loa}
\printacronyms[include=acronimos,name=\listadesiglasname,heading=chapter*]
\cleardoublepage

% Sumário
\pdfbookmark[0]{Table of contents}{toc}
\tableofcontents*
\cleardoublepage

\textual

% Introdução - descrever projeto em cooperação (front/back), extenso e feito em 2 mãos, estabelecer limites
% Motivação - SAP (unipampa), gestão da extensão, inexistência da gestão do projeto no sistema (emissão de certificados...) 
%==============================================================================
\chapter{Introduction}\label{introduction}
%==============================================================================

This work is part of a collaborative effort by two students from the Software Engineering course. Since the complexity and size of the problem were bigger than what the academy is used to seeing on term papers, the work was split among both authors. This decision was supported and previously agreed upon by their supervisor.

The effort was separated as follows: While this paper encompasses all of the front-end system requirements, such as analytics, multiple languages, component styling, design of the pages with the user interface and user experience, the counterpart focuses heavily on the back-end system requirements. Both projects are separate implementations and live in different version control repositories, and both have their own specific devops pipelines and deployments.

The \acl{UNIPAMPA} offers several opportunities for students to participate in environments external to the university. According to the 317th CONSUNI Resolution from April 29th, 2021, an outreach activity can be described as the following: An action that integrates the curricular matrix and the organization of research, constituting an interdisciplinary, political, educational, cultural, scientific and technological process. It also promotes the transforming interaction between \ac{UNIPAMPA} and society, through the production and application of knowledge, in permanent articulation with teaching and research \cite{res317}.

There are 4 different modalities for outreach activities \cite{res317}:
\begin{inparaenum}[(i)]
    \item Program: a set of actions that are oriented towards a common objective, with a medium to long term duration;
    \item Project: usually linked to a Program, it has a specific objective and a fixed term;
    \item Course: training activity, with short duration, and;
    \item Event: an action with an artistic, cultural and scientific character, with a well-defined duration.
\end{inparaenum}

An example is the JEDI Program, which aims to solve local problems and stimulate capacity building and training in \acl{IT} (\ac{IT}) with the involvement of the community (academic and external) together with public or private companies \cite{chamadaJedi}.

To register a new \acl{OCA}, it is first necessary to identify whether it is a Specific or Linked \ac{OCA} - whether it is linked to an Undergraduate Curriculum Component or not \cite{res317}. The \ac{OCA} insertion process is carried out at the \acl{PROEXT} (\ac{PROEXT}) of Unipampa \cite{res317}. Once registered, the course committee will need to appoint one or more professors as outreach supervisors \cite{res317}.

Among the supervisor's responsibilities are: the evaluation of the formative nature of the action carried out by the student, the validation of the use of Specific \acp{OCA} and also the construction and dissemination of a biannual report containing the extension activities carried out in the course.

After contacting the supervisor, showing interest in an \ac{OCA}, it is the student's responsibility to request the use and validation of the hours spent in the activity with the Academic Secretary of the course \cite{res317}. And the professor is responsible for selecting and enrolling each student interested in the \ac{OCA}, until there are open slots.

%------------------------------------------------------------------------------
\section{Motivation}\label{sec:motivation}
%------------------------------------------------------------------------------

It's not a mystery that time is of utmost importance on the academic environment. It is an invaluable resource, and as such, must be dealt with with great care. Thinking about time is what drives this project forward, as currently, there is no solution to take care of all the requirements of creating and managing outreach activities in \ac{UNIPAMPA}.

In 2023, due to Res. Nº317 \cite{res317}, the process of curricularization of new \acl{OCA} will be obligatorily implemented by universities in Brazil. However, all management would be carried out manually by the coordinator or collaborators of the Outreach Programs and Projects. With that in mind, a number of issues were identified with this manual approach that would be easily resolved by introducing a tool to support the process.

This means that everything - from developing a project, submitting and having it approved, sending emails and creating registration forms to open it for the students to join and later on receive their participation certificate - has to be manually done by the professors and coordinators. From the student's perspective, there is a possibility that one or more of the offers will go unnoticed amid the large amount of emails received daily from the university. The whole process is unoptimized, and takes a great amount of time and effort to be concluded.

So in order to create a more efficient and welcoming environment for the outreach activities in the university, the idea of a system to support the needs of this whole process was conceived.

Also due to the institutional action ``Unipampa Cidadã'' - which aims to dedicate a portion of the hours currently invested in outreach activities in projects and areas of great social relevance - it is expected that the enrollment rate of new students in higher education will increase \cite{unipampacidada}, which consequently highlights even more the importance of automating manual processes at the university.

%------------------------------------------------------------------------------
\section{Objectives}\label{sec:objectives}
%------------------------------------------------------------------------------

According to what has been presented, this Course Conclusion Work has the general objective of developing the front-end part of a tool in which all the current management of \acp{OCA} will be carefully observed and reproduced, in order to reduce the effort of the professors and supervisors with the manual steps of the process.

In order to achieve the general objective, the following specific objectives were defined:

\begin{itemize}
    \item Systematically review grey literature works and products in order to find similar solutions, collecting the first batch of requirements.
    \item Elaborate a survey, according to \cite{kasunic2005designing}, in order to discover new system requirements and in order to better understand the target users' needs.
    \item Analyze the results and refine the elicited requirements to create tangible tasks and an implementation roadmap.
    \item Study current market technologies, programming languages and frameworks to build a stack which delivers a great user experience and is creates a codebase that is easily maintained.
    \item Create a working \acl{MVP} (\ac{MVP}) of the system which implements at first the most critical collected and refined requirements.
\end{itemize}

%------------------------------------------------------------------------------
\section{Contribution}\label{sec:contribution}
%------------------------------------------------------------------------------

The main contribution of this study is the implementation of an \ac{MVP}, in the form of a web application, to support and automate the whole process of \aclp{OCA} in the university. Due to the complexity of this proposal, as previously mentioned, the effort was split amongst two papers. This focuses on the development of a web app, with all its related challenges, but it doesn't encompasses the backend services in detail.

As for the artifacts generated to support the research, such as the gray literature systematic review and the survey, all of them were done in conjunction by both authors and are not related specifically to a single work.

%------------------------------------------------------------------------------
\section{Organization}\label{sec:organization}
%------------------------------------------------------------------------------

This document is organized according the following chapters:

\begin{itemize}
    \item \textbf{Chapter 2: Methodology}: Describes how the study was planned and the approaches used to conduct it.
    \item \textbf{Chapter 3: Background}: Important information and details of concepts related to the study, e.g. outreach activities in Brazil and in the \acl{UNIPAMPA}, federal laws and similar tools.
    \item \textbf{Chapter 4: Gray Literature}: How the protocol was structured, results, discovered tools, preliminary requirements.
    \item \textbf{Chapter 5: Survey}: How it was structured, results, validation of refined requirements with the target audience.
    \item \textbf{Chapter 6: Extensionly}: Revolves around implementation details, created artifacts, technologies used, the software engineering process, DevOps practices and the incorporation of analytics.
\end{itemize}

% Este trabalho mostra alguns exemplos de utilização de comandos \LaTeX, opções de formatação e dicas de conteúdo.
%   Várias partes foram retiradas do manual da classe \abnTeX~\cite{abntex2classe,abntex2cite}, e algumas partes, principalmente o anexo \ref{anexo:latex}, de \cite{Moro2012}.
%   A propósito, recomenda-se a leitura de \cite{Moro2012}, pois contém dicas de como escrever um trabalho de pós-graduação que podem ser aplicadas também a tcc.
%   Leia também \cite{SisbiUnipampa2011} para mais informações sobre como escrever um trabalho.




% Esta seção testa o uso de divisões de documentos. Isto é uma seção.


% %------------------------------------------------------------------------------
% \subsection{Divisões do documento: subseção}
% %------------------------------------------------------------------------------

% Isto é uma subseção.


% %------------------------------------------------------------------------------
% \subsubsection{Divisões do documento: subsubseção}
% %------------------------------------------------------------------------------

% Isto é uma subsubseção.


% %------------------------------------------------------------------------------
% \subsubsection{Divisões do documento: subsubseção}
% %------------------------------------------------------------------------------

% Isto é outra subsubseção.


% %------------------------------------------------------------------------------
% \subsection{Divisões do documento: subseção}\label{sec:exemplo-subsec}
% %------------------------------------------------------------------------------

% Isto é uma subseção.


% %------------------------------------------------------------------------------
% \subsubsection{Divisões do documento: subsubseção}
% %------------------------------------------------------------------------------

% Isto é mais uma subsubseção da \autoref{sec:exemplo-subsec}.


% %------------------------------------------------------------------------------
% \section{Este é um exemplo de nome de seção longo. Ele deve estar alinhado à esquerda e a segunda e demais linhas devem iniciar logo abaixo da primeira palavra da primeira linha}
% %------------------------------------------------------------------------------

% Isso atende à norma \citeonline[seções de 5.2.2 a 5.2.4]{NBR14724:2011} e \citeonline[seções de 3.1 a 3.8]{NBR6024:2012}.


% %------------------------------------------------------------------------------
% \section{Consulte o manual da classe \textsf{abntex2}}
% %------------------------------------------------------------------------------

% Consulte o manual da classe \textsf{abntex2} \cite{abntex2classe} para uma referência completa das macros e ambientes disponíveis.
%   Além disso, o manual possui informações adicionais sobre as normas ABNT observadas pelo \abnTeX.


% %------------------------------------------------------------------------------
% \section{Organização deste trabalho}
% %------------------------------------------------------------------------------

% No \autoref{desenvolvimento} há várias instruções e dicas de uso deste modelo, e o \autoref{anexo:latex}\todo{BUG do abntex2: deveria referenciar como Anexo.}~traz dicas sobre o uso do \LaTeX.
 % [OBRIGATORIO]
\input{textuais/metodologia} % [OBRIGATORIO]
% Background - extensão universitária no Brasil, curricularização da extensão, soluções/ferramentas de apoio à extensão, leis federais, resoluções unipampa, implantação da extensão, tipos de extensão, perfis de pessoas envolvidas na extensão, programas e projetos de extensão na unipampa
    % Unipampa Cidadã
%==============================================================================
\chapter{Background}\label{background}
%==============================================================================
% Literatura Cinza - protocolo, resultados, lista de ferramentas encontradas -> lista preliminar de requisitos
% Screenshot / análise / resumo sobre 
%==============================================================================
\chapter{Gray Literature}\label{grayliterature}
%==============================================================================
% Survey - lista preliminar da literatura -> validação com usuários reais, agregando novos requisitos
    % protocolo, questionário, requisitos propostos, resultados, análise
    
%==============================================================================
\chapter{Survey}\label{survey}
%==============================================================================
% Extensionly - análise e projeto de software, artefatos da implementação, maior capítulo de todos, modelo de domínio, diagrama de componentes, paradigma de programação, tecnologias, processo da engenharia de software, separação frontend/backend (com mais detalhes técnicos), usar figuras e modelos
    % seção devops
    % seção analytics
%==============================================================================
\chapter{Extensionly}\label{extensionly}
%==============================================================================
% Considerações Preliminares - possível publicação em eventos da área, com os resultados encontrados (ERES agosto), mais eventos (SBIE), artigos relacionados à extensão
%==============================================================================
\chapter{Preliminary Considerations}\label{conclusao}
%==============================================================================

Em Trabalhos de Conclusão de Curso, use ``\emph{Considerações Finais}'' e não ``\emph{Conclusão}''.

Bom trabalho!
 % [OBRIGATORIO]


% ELEMENTOS PÓS-TEXTUAIS
\postextual

\bibliography{bibliografia,abntex2-modelo-references} % Bibliografia [OBRIGATORIO]
\begin{apendicesenv}

  % % Imprime uma página indicando o início dos apêndices
  \partapendices
  % % Para cada apêndice, um \chapter


  \chapter{Translated Survey Questionnaire}\label{appendix:questionnaire}

  \includepdf[pages={1-23}]{artifacts/5-questionnaire-en.pdf}

  % De acordo com a ABNT:

  % \begin{quotation}
  % Apêndice (opcional): texto utilizado quando o autor pretende complementar sua argumentação. São identificados por letras maiúsculas e travessão, seguido do título. Ex.: APÊNDICE A - Avaliação de células totais aos quatro dias de evolução

  % Anexo (opcional): texto ou documento \textbf{não elaborado pelo autor} para comprovar ou ilustrar. São identificados por letras maiúsculas e travessão, seguido do título. Ex.: ANEXO A - Representação gráfica de contagem de células
  % \end{quotation}

  % Tais definições (e outras) podem ser encontradas na NBR 14724-2001 Informação e documentação - trabalhos acadêmicos\footnote{http://www.firb.br/abntmonograf.htm}.


  % %==============================================================================
  % \chapter{Segundo Apêndice}
  % %==============================================================================

  % Pode ser que tenha outro...


\end{apendicesenv}
 % Apêndices [OPCIONAL]
\input{postextuais/anexos} % Anexos [OPCIONAL]
\printindex % Índice Remissivo [OPCIONAL]

\end{document}

\begin{resumo}
  \textbf{Contexto:} De acordo com a Resolução nº 7 de 2018, que foi divulgada pelo Conselho Nacional de Educação (CNE) \cite{Resolucao-MEC:2018}, a curricularização da extensão se tornará um requisito obrigatório no ano de 2023. Como consequência disso, todos os programas de pós-graduação sediados em universidades federais são obrigados a destinar dez por cento de todo o tempo de aula para atividades de extensão. Além disso, os alunos estarão mais motivados a procurar ações extensionistas e os professores mais motivados a propô-las. \textbf{Objetivo:} Dado que a criação e manutenção de atividades de extensão são processos burocráticos, demorados e manuais, o objetivo deste trabalho é fornecer uma ferramenta baseada na web que possa suportar todos ou a grande maioria desses processos. Seu desenvolvimento está sendo realizado por dois alunos de graduação, com esforços direcionados igualmente entre o front-end e back-end no que diz respeito ao processo de engenharia de software. Este trabalho foca no front-end. \textbf{Metodologia:} Para tornar isso possível, duas abordagens científicas foram utilizados: a primeira é uma revisão sistemática da literatura cinza, para buscar ferramentas comparáveis e uma coleção de requisitos e detalhes pertinentes. A segunda foi um levantamento com potenciais usuários do sistema que fazem parte da comunidade acadêmica da UNIPAMPA. O objetivo era classificar as funcionalidades de acordo com a importância e, ao mesmo tempo, permitir que os participantes oferecessem sugestões relacionadas à funcionalidades existentes ou ideias totalmente novas. \textbf{Resultados:} Com os resultados obtidos, foi possível criar uma lista de requisitos que foram priorizados juntamente com funcionalidades adicionais que os respondentes do levantamento sugeriram. Com isso, o desenvolvimento da ferramenta já foi direcionado e pode começar. \textbf{Conclusões Preliminares:} Em relação à hipótese levantada por este estudo, ainda não é possível validá-la ou refutá-la porque a ferramenta ainda não foi desenvolvida e exaustivamente testada com usuários. No entanto, dados os resultados animadores do levantamento e a revisão da literatura cinza, é bastante provável que a hipótese seja confirmada após o desenvolvimento preciso e completo da aplicação final.

  \vspace{\onelineskip}

  \noindent
  \textbf{Palavras-chave}: Aplicação web. Survey. Literatura Cinza. Front-end. Extensão. Universidade. NextJS. React.
\end{resumo}

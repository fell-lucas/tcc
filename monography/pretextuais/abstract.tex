\begin{resumo}[Abstract]
  \textbf{Context:} In accordance with the Resolution No. 7 of 2018, which was released by the National Education Council (CNE) \cite{Resolucao-MEC:2018}, the curricularization of outreach activities will turn into a mandatory requirement in the year 2023. As a consequence of this, all undergraduate programs housed inside federal universities are required to allocate ten percent of the entire amount of class time to the outreach activities. Because of this, students will be more motivated to look for more outreach events, and teachers will be more inspired to propose them. \textbf{Objective:} Given that the creation and maintenance of outreach activities are bureaucratic, time-consuming, and manual processes, the goal of this work is to provide a web-based tool that can support all or the vast majority of these processes. Its development is being carried out by two undergraduate students, with equal effort being put forward to the front-end and back-end with regard to the software engineering process. This work is concerned with the front-end. \textbf{Methodology:} To make this possible, two scientific methods were used: the first is a systematic review of grey literature, to look for comparable tools and a collection of pertinent requirements and details. The second was a survey with potential system users that are a part of the UNIPAMPA academic community. The goal was to rank the functionalities according to importance while also allowing participants to offer suggestions that were either related to the existing features or brand-new. \textbf{Results:} With the results obtained, it was possible to create a list of requirements that were prioritized together with additional features that the survey respondents suggested. As a result, the development of the tool has already been directed and can begin. \textbf{Preliminary Considerations:} Regarding the hypothesis raised by this study, it is not yet possible to validate or refute it because the tool has not yet been developed and thoroughly tested with users. However, given the encouraging results from the survey and the review of the grey literature, it is quite probable that the hypothesis will be confirmed after the accurate and thorough development of the final application.

  \vspace{\onelineskip}

  \noindent
  \textbf{Key-words}: Web application. Survey. Grey Literature. Front-end. Outreach Activities. University. NextJS. React.
\end{resumo}
